\chapter[Lie Groups as Spin Groups]{Lie Groups as Spin Groups\footnote{This chapter follows \cite{DHS&VA}}}

\section{Introduction}

A Lie group, $G$, is a group that is also a differentiable manifold. So that if $g \in G$ and $x \in \Re^{N}$ then
there is a differentiable mapping $\phi:\Re^{N}\rightarrow G$ so that for each $g \in G$ we can define a tangent 
space $\mathcal{T}_{g}$.

A example (the one we are most concerned with) of a Lie group is the group of $n\times n$ non-singular matrices.  The 
coordinates of the manifold are simply the elements of the matrix so that the dimension of the group manifold is $n^{2}$
and the matrix is obviously a continuous differentiable function of its coordinates.

A Lie algebra is a vector space $\mfrk{g}$ with a bilinear operator $\mat{\cdot,\cdot}:\mfrk{g}\times\mfrk{g}\rightarrow\mfrk{g}$
called the \emph{Lie bracket} which satifies ($x,y,z \in\mfrk{g}$ and $\alpha,\beta\in\Re$):
\begin{align}
	\mat{ax+by,z} &=a\mat{x,z}+b\mat{y,z}, \label{la1ax}\\
	\mat{x,x} &= 0, \label{la2ax}\\
	\mat{x,\mat{y,z}}+\mat{z,\mat{x,y}}+\mat{y,\mat{z,x}} &= 0\label{la3ax}.
\end{align}
Equations~(\ref{la1ax}) and (\ref{la2ax}) imply $\mat{x,y} = -\mat{y,x}$, while eq~(\ref{la3ax}) is the Jacobi identity.

The purpose of the following analysis is to show that the Lie algebra of the general linear group of dimension $n$
over the real numbers (the group of $n\times n$ invertible matrices), $GL\paren{n,\Re}$, can be represented by a rotation
group in an appropriate vector space, $\mathcal{M}$ (let $\paren{p,q}$ be the signature of the new vector space). Furthermore, 
since the rotation group, $SO\paren{p,q}$ can be represented by the spin group, $Spin\paren{p,q}$, in the same vector space. Then by use of
geometric algebra we can construct any rotor, $R\in Spin\paren{p,q}$, as $R = e^{\frac{B}{2}}$ where $B$ is a bi-vector in the geometric 
algebra of $\mathcal{M}$ and the bi-vectors form a Lie algebra under the commutator product.\footnote{Since we could be dealing with
vector spaces of arbitrary signature $\paren{p,q}$ we use the following nomenclature:
\begin{center}\begin{tabular}{cl}
	$\f{SO}{p,q}$  & Special orthogonal group in vector space with signature $\paren{p,q}$ \\
	$\f{SO}{n}$  & Special orthogonal group in vector space with signature $\paren{n,0}$ \\
	$\f{Spin}{p,q}$  & Spin group in vector space with signature $\paren{p,q}$ \\
	$\f{Spin}{n}$  & Spin group in vector space with signature $\paren{n,0}$ \\
	$\f{GL}{p,q,\Re}$  & General linear group in real vector space with signature $\paren{p,q}$ \\
	$\f{GL}{n,\Re}$  & General linear group in real vector space with signature $\paren{n,0}$
\end{tabular}\end{center}}

The main trick in doing this is to construct the appropriate vector space, $\mathcal{M}$, with subspaces isomorphic to $\Re^{n}$
so that a rotation in $\mathcal{M}$ is equivalent to a general linear transformation in a subspace of $\mathcal{M}$ isomorphic 
to $\Re^{n}$.  We might suspect that $\mathcal{M}$ cannot be a Euclidean space since a general linear transformation can cause
the input vector to grow as well as shrink.

\section{Simple Examples}

\subsection{$\f{SO}{2}$ - Special Orthogonal Group of Order 2}

The group is represented by all $2\times 2$ real matrices\footnote{We denote linear transformations with an underbar.} $\ubar{R}$ where\footnote{In this case $\ubar{I}$ is the identity matrix and not the
pseudo scalar.} $\ubar{R}\ubar{R}^{T} = \ubar{I}$ and $\det\paren{\ubar{R}} = 1$.  The group product is
matrix multiplication and it is a group since if $\ubar{R}_{1}\ubar{R}_{1}^{T} = \ubar{I}$ and $\ubar{R}_{2}\ubar{R}_{2}^{T} = \ubar{I}$ then
\begin{align}
	 \paren{\ubar{R}_{1}\ubar{R}_{2}}\paren{\ubar{R}_{1}\ubar{R}_{2}}^{T} &= \ubar{R}_{1}\ubar{R}_{2}\ubar{R}_{2}^{T}\ubar{R}_{1}^{T} = \ubar{R}_{1}\ubar{I}\ubar{R}_{1}^{T} = \ubar{R}_{1}\ubar{R}_{1}^{T} = \ubar{I} \\
	 \f{\det}{\ubar{R}_{1}\ubar{R}_{2}} &= \f{\det}{\ubar{R}_{1}}\f{\det}{\ubar{R}_{2}} = 1.
\end{align}
$SO\paren{2}$ is also a Lie group since all $\ubar{R}\in\f{SO}{2}$ can be represented as a continuous function of the coordinate $\theta$:
\begin{equation}
	\f{\ubar{R}}{\theta} = \mat{
	\begin{array}{cc}
		\f{\cos}{\theta} & -\f{\sin}{\theta} \\
		\f{\sin}{\theta} & \f{\cos}{\theta}
	\end{array}
	}.
\end{equation}
In this case the spin representation of $\f{SO}{2}$ is trivial, namely\footnote{For multivectors such as the rotor $R$ there is no 
underbar. We have $\f{\ubar{R}}{a} = \ubar{R}a = RaR^{\R}$ where $a$ is a vector.}
$\f{R}{\theta} = e^{\frac{I}{2}\theta} = \f{\cos}{\frac{\theta}{2}} + I\f{\sin}{\frac{\theta}{2}}$, 
$\f{R}{\theta}^{\R} = e^{-\frac{I}{2}\theta} = \f{\cos}{\frac{\theta}{2}} - I\f{\sin}{\frac{\theta}{2}}$ where $I$
is the pseudo-scalar for $\f{\mathcal{G}}{\Re^{2}}$. Then we have $\f{\ubar{R}}{\theta}a = \f{R}{\theta}a\f{R}{\theta}^{\R}$.

\subsection{$\f{GL}{2,\Re}$ - General Real Linear Group of Order 2}

The group is represented by all $2\times 2$ real matrices $\ubar{A}$ where $\f{\det}{\ubar{A}} \ne 0$.  Again the the group product is
matrix multiplication and it is a group because if $\ubar{A}$ and $\ubar{B}$ are $2\times 2$ real matrices then 
$\f{\det}{\ubar{A}\ubar{B}} = \f{\det}{\ubar{A}}\f{\det}{\ubar{B}}$ and if $\f{\det}{\ubar{A}}\ne 0$ and $\f{\det}{\ubar{B}}\ne 0$ then $\f{\det}{\ubar{A}\ubar{B}}\ne 0$.  Any member of 
the group is represented by the matrix $\paren{a = \paren{a^{1},a^{2},a^{3},a^{4}}\in \Re^{4}}$:
\begin{equation}
	\f{\ubar{A}}{a} = \mat{
	\begin{array}{cc}
		a^{1} & a^{2} \\
		a^{3} & a^{4}
	\end{array}
	}.
\end{equation}
Thus any element $\ubar{A}\in \f{GL}{2,\Re}$ is a continuous function of $a = \paren{a^{1},a^{2},a^{3},a^{4}}$.  Thus $\f{GL}{2,\Re}$ is a four
dimensional Lie group while $\f{SO}{2}$ is a one dimensional Lie group.

Another difference is that  $\f{SO}{2}$ is compact while $\f{GL}{2,\Re}$ is not.  $\f{SO}{2}$ is compact since for any
convergent sequence $\paren{\theta_{i}}$,
\begin{equation*}
	\lim_{i\rightarrow\infty}\f{\ubar{R}}{\theta_{i}}\in \f{SO}{2}.
\end{equation*} 
$\f{GL}{2,\Re}$ is not compact since there is at least one convergent sequence $\paren{a_{i}}$ 
such that
\begin{equation*}
	\lim_{i\rightarrow\infty}\f{\det}{\f{\ubar{A}}{a_{i}}} = 0.
\end{equation*} 
After we develop the required theory we will calculate the spin representation of $\f{GL}{2,\Re}$.

\section{The Grassmann Algebra}

Let $\mathcal{V}^{n}$ be an $n$-dimensional real vector space with basis $\set{\wb_{i}}: 1\le i \le n$ and $\W$ is the
outer (wedge) product for the geometric algebra define on $\mathcal{V}^{n}$. As before the geometric object
\be
	v_{1}\W v_{2}\W\dots\W v_{k}
\ee
is a $k$-blade in the Grassmann or the geometric algebra where the $v_{i}$'s are $k$ independent vectors in
$\mathcal{V}_{n}$ and a linear combination of $k$-blades is called a $k$-vector.  The space of all $k$-vectors is
just all the $k$-grade multivectors in the geometric algebra $\GA{\mathcal{V}_{n}}$. Denote the space of all
$k$-vectors in $\mathcal{V}^{n}$ by $\Lambda_{n}^{k} = \f{\Lambda^{k}}{\mathcal{V}^{n}}$ with
\be
	\f{\dim}{\Lambda_{n}^{k}} = \binom{n}{k} = \bfrac{n!}{k!\paren{n-k}!}.
\ee
Letting $\Lambda_{n}^{1} = \mathcal{V}^{n}$ and $\Lambda_{n}^{0} = \Re$ the entire Grassmann algebra is a 
$2^{n}$-dimensional space.\footnote{Consider the geometric algebra $\GA{\mathcal{V}^{n}}$ of $\mathcal{V}^{n}$
with a null basis $\set{\wb_{i}}:\wb_{i}^{2}=0,1\le i \le n$. Then 
\begin{align*}
	\paren{\wb_{i}+\wb_{j}}^2 &= \wb_{i}^{2}+\wb_{i}\wb_{j}+\wb_{j}\wb_{i}+\wb_{j}^{2} \\
	                        0 &= \wb_{i}\wb_{j}+\wb_{j}\wb_{i} \\
	                        \wb_{i}\wb_{j} &= -\wb_{j}\wb_{i} \\
	                        0 &= 2\wb_{i}\cdot\wb_{j}.
\end{align*}
If the basis set is null the metric tensor is null and $\wb_{i}\wb_{j} = \wb_{i}\W\wb_{j}$ even if $i=j$ and the geometric
algebra is a Grassmann algebra.}
\be
	\Lambda_{n} = \sum_{k=0}^{n} \Lambda_{n}^{k}.
\ee
The Grassmann algebra of a vector space $\mathcal{V}^{n}$ is denoted by $\f{\Lambda}{\mathcal{V}^{n}}$.


\section{The Dual Space to $\mathcal{V}_{n}$}

The dual space $\mathcal{V}^{n*}$ to $\mathcal{V}^{n}$ is defined as follows.  Let $\set{\wb_{i}}$ be a
basis for $\mathcal{V}^{n}$ and define the basis, $\set{\wb_{i}^{*}}$ for $\mathcal{V}^{n*}$ by
\be\label{eq8_4}
	\wb_{i}^{*}\cdot \wb_{j} = \half\delta_{ij}. 
\ee
Again the dual space $\mathcal{V}^{n*}$ has its own Grassmann algebra $\f{\Lambda^{*}}{\mathcal{V}^{n}}$
given by
\be
	\f{\Lambda^{*}}{\mathcal{V}^{n}} = \Lambda^{*}_{n} = \sum_{k=0}^{n}\Lambda_{n}^{k*}.
\ee
$\f{\Lambda^{*}}{\mathcal{V}^{n}}$ can be represented as a geometric algebra by imposing the null metric condition
$\paren{\wb_{i}^{*}}^{2}=0$ as in the case of $\f{\Lambda}{\mathcal{V}^{n}}$. 

\section{The Mother Algebra}

From the base vector space and it's 
dual space one can construct a $2n$ dimensional vector space from the direct sum of the two vector 
spaces as defined in Appendix~\ref{DSVS}
\be\label{eq8_6}
	\Re^{n,n} \equiv \mathcal{V}^{n}\oplus \mathcal{V}^{n*}
\ee
with basis $\set{\wb_{i},\wb_{j}^{*}}$.  An orthogonal basis for $\Re^{n,n}$ can be constructed as follows (using
equation~\ref{eq8_4}):
\begin{align}
	\eb_{i} &= \wb_{i}+\wb_{i}^{*} \label{eq8_7}\\
	\ebb_{i} &= \wb_{i}-\wb_{i}^{*} \label{eq8_7a}\\
	\eb_{i}\cdot\eb_{j} &= \paren{\wb_{i}+\wb_{i}^{*}}\cdot\paren{\wb_{j}+\wb_{j}^{*}} \nonumber \\
	                    &=  \wb_{i}\cdot\wb_{j}+\wb_{i}\cdot\wb_{j}^{*}+\wb_{i}^{*}\cdot\wb_{j}+
	                        \wb_{i}^{*}\cdot\wb_{j}^{*} \nonumber \\
	                    &= \delta_{ij} \label{eiej}\\
	\ebb_{i}\cdot\ebb_{j} &= \paren{\wb_{i}-\wb_{i}^{*}}\cdot\paren{\wb_{j}\wb_{j}^{*}} \nonumber \\
	                    &=  \wb_{i}\cdot\wb_{j}\wb_{i}\cdot\wb_{j}^{*}-\wb_{i}^{*}\cdot\wb_{j}+
	                        \wb_{i}^{*}\cdot\wb_{j}^{*} \nonumber \\
	                    &= -\delta_{ij} \label{beibej}\\
	\eb_{i}\cdot\ebb_{j} &= \paren{\wb_{i}+\wb_{i}^{*}}\cdot\paren{\wb_{j}-\wb_{j}^{*}} \nonumber \\
	                    &=  \wb_{i}\cdot\wb_{j}-\wb_{i}\cdot\wb_{j}^{*}+\wb_{i}^{*}\cdot\wb_{j}-
	                        \wb_{i}^{*}\cdot\wb_{j}^{*} \nonumber \\
	                    &= 0. \label{eibej}	                    
\end{align}
Thus $\Re^{n,n}$ can also be represented by the direct sum of an $n$-dimensional Euclidian vector space, $E^{n}$,
and a $n$-dimensional anti-Euclidian vector space (square of all basis vectors is $-1$) $\bar{E}^{n}$
\be\label{eq8_12}
	\Re^{n,n} = E^{n}\oplus \bar{E}^{n}.
\ee
In this case $E^{n}$ and $\bar{E}^{n}$ are orthogonal\footnote{Every vector in $E^{n}$
is orthogonal to every vector in $\bar{E}^{n}$.} since $\eb_{i}\cdot\bar{\eb}_{j} = 0$.
The geometric algebra of $\Re^{n,n}$ is defined and denoted by
\be
	\Re_{n,n} \equiv \GA{\Re^{n,n}}
\ee
has dimension $2^{2n}$ with $k$-vector subspaces $\Re_{n,n}^{k}=\f{\mathcal{G}^{k}}{\Re^{n,n}}$ and is 
called the {\em mother algebra}.\footnote{ 
As an example of the equivalence of $E^{n}\oplus \bar{E}^{n}$ and $\mathcal{V}^{n}\oplus \mathcal{V}^{n*}$ consider
the metric tensors of each representation of $\Re^{2,2}$. The metric tensor of $E^{2}\oplus \bar{E}^{2}$ is
\begin{equation*}
	\mat{
		\begin{array}{cccc}
			1 & 0 & 0 & 0 \\
			0 & 1 & 0 & 0 \\
			0 & 0 & -1 & 0 \\
			0 & 0 & 0 & -1 
		\end{array}}
\end{equation*}
with eigenvalues $\mat{1,1,-1,-1}$.  Thus the signature of the metric is $\paren{2,2}$. The metric tensor of
$\mathcal{V}^{2}\oplus \mathcal{V}^{2*}$ is 
\begin{equation*}
	\mat{
		\begin{array}{cccc}
			0 & 0 & 1 & 0 \\
			0 & 0 & 0 & 1 \\
			1 & 0 & 0 & 0 \\
			0 & 1 & 0 & 0 
		\end{array}}/2
\end{equation*}
with eigenvalues $\mat{1/2,1/2,-1/2,-1/2}$.  Thus the signature of the metric is $\paren{2,2}$.  As expected both
representations have the same signature.} 

From the basis $\set{\eb_{i},\ebb_{i}}$ we can construct $\paren{p+q}$-blades
\be\label{eq8_14}
	E_{p,q} = E_{p}\W\bar{E}_{q}^{\R} = E_{p}\bar{E}_{q}^{\R},
\ee 
where\footnote{Since the $\set{\eb_{i},\ebb_{i}}$ form an orthogonal set and specifying that no factors are repeated 
we can use the geometric product in 
equations~\ref{eq8_15} and \ref{eq8_16} instead of the outer (wedge) product.}
\begin{align}
	E_{p} &= \eb_{i_{1}}\eb_{i_{2}}\dots\eb_{i_{p}} = E_{p,0}\;\; 1\le i_{1}< i_{2}<\dots < i_{p}\le n \label{eq8_15}\\
	\bar{E}_{q} &= \ebb_{j_{1}}\ebb_{j_{2}}\dots\ebb_{j_{q}} = \bar{E}_{0,q}\;\; 1\le j_{1}< j_{2}
	              <\dots < j_{q}\le n.\label{eq8_16}
\end{align}
Each blade determines a projection of $\ubar{E}_{p,q}$ of $\Re^{n,n}$ into a $\paren{p+q}$-dimensional subspace
$\Re^{p,q}$ defined by (see section~\ref{sec5_1_2} and equation~\ref{24})\footnote{The underbar notation as in 
$\f{\ubar{E}_{p,q}}{a}$ allows one to distinguish linear operators from elements in the algebra such as $E_{p,q}$.}
\be\label{eq8_17}
	\ubar{E}_{p,q}a \equiv \paren{a\cdot E_{p,q}}E_{p,q}^{-1} = \half\paren{a-\paren{-1}^{p+q}E_{p,q}aE_{p,q}^{-1}}. 
\ee
A vector, $a$, is in $\Re^{p,q}$ if and only if 
\be
	a\W E_{p,q} = 0 = aE_{p,q}+\paren{-1}^{p+q}E_{p,q}a.
\ee
For $p+q=n$, the blade $E_{p,q}$ determines a split of $\Re^{n,n}$ into orthogonal subspaces with {\em complementary 
signature}\footnote{The signature of $\Re^{p,q}$ is $(p,q)$ as opposed to $\bar{\Re}^{p,q}$ with signature $(q,p)$.}, 
as expressed by
\be
	\Re^{n,n} = \Re^{p,q}\oplus\bar{\Re}^{p,q}.
\ee  
For the case of $q=0$ equation~\ref{eq8_17} can be written as
\be
	\ubar{E}_{n}a = \half\paren{a+a^{*}},
\ee
where $a^{*}$ is defined by
\be
	a^{*} \equiv \paren{-1}^{n+1}E_{n}aE_{n}^{-1}.
\ee
It follows immediately that $\eb_{i}^{*} = \eb_{i}$ and $\paren{\bar{\eb}_{i}}^{*} = -\bar{\eb}_{i}$.\footnote{This is
obvious by
\begin{align*}
	\eb_{i}^{*} &= \paren{-1}^{n+1}E_{n}\eb_{i}E_{n}^{-1} = \paren{-1}^{n+1}\paren{-1}^{n-1}\eb_{i}E_{n}E_{n}^{-1} 
	           = \paren{-1}^{2n}\eb_{i} = \eb_{i}, \\
	\paren{\bar{\eb}_{i}}^{*}  &= \paren{-1}^{n+1}E_{n}\paren{\bar{\eb}_{i}}E_{n}^{-1} =
	                             \paren{-1}^{n+1}\paren{-1}^{n}\paren{\bar{\eb}_{i}}E_{n}E_{n}^{-1} 
	                          = \paren{-1}^{2n+1}\paren{\bar{\eb}_{i}} = -\paren{\bar{\eb}_{i}}.
\end{align*}
}
The split of $\Re^{n,n}$ given by equation~\ref{eq8_6} cannot be constructed as the split in equation~\ref{eq8_12} since 
the vectors expanded in the $\set{\wb_{i},\wb_{j}^{*}}$ basis cannot be normalized since they are null vectors. Instead
consider a bivector\footnote{In general the basis blades for bivectors in $\Re_{n,n}$ do not commute since
the dimension of the bivector subspace, $\Re_{n,n}^{2}$ is 
\begin{equation*}
	\binom{2n}{2} = \frac{\paren{2n}!}{2!\paren{2n-2}!} = n\paren{2n-1} = 2n^{2}-1.
\end{equation*}
If one has more than $n$ bivector blades the planes defined by at least two of the blades will intersect.} $K$ in $\Re^{n,n}$
\be
	K = \sum_{i=0}^{n} K_{i},\label{eq8_22}
\ee
where the $K_{i}$ are distinct commuting blades, $K_{i}\times K_{j} = 0$ (commutator product), normalized to $K_{i}^{2}=1$.  
The bivector $K$ defines
the automorphism $\ubar{K}:\Re^{n,n}\rightarrow\Re^{n,n}$
\be
	\bar{a} = \ubar{K}a \equiv a\times K = a\cdot K.
\ee
This maps every vector $a$ into the vector $\bar{a}$ which is called the {\em complement} of $a$ with respect to $K$.

Each $K_{i}$ is a bivector blade that defines a two dimensional Minkowski (since $K_{i}^{2}=1$) subspace of $\Re^{n,n}$. 
Since the blades commute, $K_{i}\times K_{j} = 0$, they define disjoint subspaces of $\Re^{n,n}$ and since there are 
$n$ of them they span $\Re^{n,n}$. Since $K_{i}^{2}=1$ there exists an orthonormal Minkowski basis $\eb_{i}$ and
$\bar{\eb}_{i}$ such that
\begin{align}
	\eb_{i}\cdot\bar{\eb}_{j} &= 0,\;i\ne j, \label{eq8_24a}\\
	\eb_{i}\cdot\eb_{i} &= -\bar{\eb}_{i}\cdot\bar{\eb}_{i} = 1. \label{eq8_25a}
\end{align}
Then if $a\in \Re^{n,n}$ and using equations~\ref{eq8_24a}, \ref{eq8_25a}, and from
appendix~\ref{app_A} equations~\ref{eq447} and \ref{B5} we have
\begin{align}
	a\cdot K_{i} &= a\cdot\paren{\eb_{i}\bar{\eb}_{i}} = 
	                -\paren{a\cdot\bar{\eb}_{i}}\eb_{i}+\paren{a\cdot\eb_{i}}\bar{\eb}_{i}, \\
	\paren{a\cdot K_{i}}\cdot K_{i} &= \paren{a\cdot\paren{\eb_{i}\bar{\eb}_{i}}}\cdot\paren{\eb_{i}\bar{\eb}_{i}} \nonumber \\
	                 &= \paren{a\cdot\eb_{i}}\eb_{i}-\paren{a\cdot\bar{\eb}_{i}}\bar{\eb}_{i} \nonumber \\
	                 &= \paren{a\cdot\eb^{i}}\eb_{i}+\paren{a\cdot\bar{\eb}^{i}}\bar{\eb}_{i} = a, \\
	\paren{a\cdot K_{i}}\cdot K_{j} &= \paren{\paren{a\cdot \bar{\eb}_{i}}\paren{\eb_{i}\cdot \bar{\eb}_{j}}
	                                  -\paren{a\cdot \eb_{i}}\paren{\bar{\eb}_{i}\cdot \bar{\eb}_{j}}}\eb_{j} \\ \nonumber
	                                &  +\paren{\paren{a\cdot \eb_{i}}\paren{\bar{\eb}_{i}\cdot \eb_{j}}
	                                  -\paren{a\cdot \bar{\eb}_{i}}\paren{\eb_{i}\cdot \eb_{j}}}\bar{\eb}_{j} = 0, \;\;\forall\; i \ne j
\end{align}
Then
\be\label{eq8_29}
	\ubar{K}a = \sum_{i=1}^{n}a\cdot K_{i} = \sum_{i=1}^{n}\paren{\paren{a\cdot\eb_{i}}\bar{\eb}_{i}-
	                  \paren{a\cdot\bar{\eb}_{i}}\eb_{i}}
\ee
and\footnote{Remember that since $K_{i}$ defines a Minkowski subspace we have for the reciprocal
basis $\eb^{i}=\eb_{i}$ and $\bar{\eb}^{i}=-\bar{\eb}_{i}$.}
\begin{align}
	\ubar{K}^{2}a &= \sum_{j=1}^{n}\paren{\sum_{i=1}^{n}a\cdot K_{i}}\cdot K_{j} \nonumber \\
	                    &= \sum_{j=1}^{n}\sum_{i=1}^{n}\paren{a\cdot K_{i}}\cdot K_{j} \nonumber \\
	                    &= \sum_{i=1}^{n}\paren{a\cdot K_{i}}\cdot K_{i} \nonumber \\
	                    &= \sum_{i=1}^{n}\paren{\paren{a\cdot\eb_{i}}\eb_{i}
	                    -\paren{a\cdot\bar{\eb}_{i}}\bar{\eb}_{i}} \nonumber \\
					    &= \sum_{i=1}^{n}\paren{\paren{a\cdot\eb^{i}}\eb_{i}
	                    +\paren{a\cdot\bar{\eb}^{i}}\bar{\eb}_{i}} \nonumber \\
	                    &= a.
\end{align}
Also
\begin{align}
	a\cdot\bar{a} &= \sum_{i=1}^{n} a\cdot\paren{a\cdot K_{i}} \nonumber \\
	              &= \sum_{i=1}^{n} a\cdot\paren{\paren{a\cdot\eb_{i}}\bar{\eb}_{i}
	                 -\paren{a\cdot\bar{\eb}_{i}}\eb_{i}} \nonumber \\
				  &= \sum_{i=1}^{n} \paren{\paren{a\cdot\eb_{i}}\paren{a\cdot\bar{\eb}_{i}}
	                 -\paren{a\cdot\bar{\eb}_{i}}\paren{a\cdot\eb_{i}}} \nonumber \\
	              &= 0,	\\             
	a^{2} &= \sum_{i=1}^{n}\paren{\paren{a\cdot \eb_{i}}^{2}-\paren{a\cdot \bar{\eb}_{i}}^{2}}, \\
	\bar{a}^{2} &= \sum_{i=1}^{n}\paren{\paren{a\cdot \bar{\eb}_{i}}^{2}-\paren{a\cdot \eb_{i}}^{2}}, \\
	a^{2}+\bar{a}^{2} &= 0.
\end{align}
Thus defining
\be
	a_{\pm} \equiv a\pm\bar{a} = a \pm \ubar{K}a = a \pm a\cdot K,
\ee
we have
\begin{align}
	\paren{a_{\pm}}^{2} &= 0, \\
	a_{+}\cdot a_{-} &= a^{2}-\bar{a}^{2} = 2\sum_{i=1}^{n}\paren{a\cdot \eb_{i}}^{2}.
\end{align}
Thus the sets $\set{a_{+}}$ and $\set{a_{-}}$ of all such vectors are in dual $n$-dimensional 
vector spaces $\left (a_{+}\in \mathcal{V}^{n}\right .$ and $\left . a_{-}\in \mathcal{V}^{n*}\right )$, so $K$ 
determines the desired null space decomposition of the form in 
equation~\ref{eq8_6} without referring to a vector basis.\footnote{The properties 
$K_{i}\times K_{j} = 0$ and $K_{i}^{2}=1$ allows us to construct $K$ from the basis 
$\set{\eb_{i},\bar{\eb}_{i}}$, but do not specify any particular basis.}

The $K$ for a given $\Re^{n,n}$ is constructed from the basis given in equations~\ref{eq8_7}
and \ref{eq8_7a}.  Then we have from equation~\ref{eq8_29}
\begin{align}
	\ubar{K}\eb_{i} = \bar{\eb}_{i}, \label{eq8_38}\\
	\ubar{K}\bar{\eb}_{i} = \eb_{i}. \label{eq8_39}
\end{align}
Also from equation~\ref{eq8_29}
\begin{align}
	\ubar{K}\wb_{i} &= \paren{\wb_{i}\cdot\eb_{i}}\bar{\eb}_{i}
	                        -\paren{\wb_{i}\cdot\bar{\eb}_{i}}\eb_{i}, \nonumber \\
	                      &= \half\paren{\paren{\paren{\eb_{i}+\bar{\eb}_{i}}\cdot\eb_{i}}\bar{\eb}_{i}
	                        -\paren{\paren{\eb_{i}+\bar{\eb}_{i}}\cdot\bar{\eb}_{i}}\eb_{i}}, \nonumber \\
	                      &= \half\paren{\bar{\eb}_{i}+\eb_{i}}, \nonumber \\
	                      &= \wb_{i}, \label{eq8_40}\\
	\ubar{K}\wb_{i}^{*} &= \paren{\wb_{i}^{*}\cdot\eb_{i}}\bar{\eb}_{i}
	                        -\paren{\wb_{i}^{*}\cdot\bar{\eb}_{i}}\eb_{i}, \nonumber \\
	                      &= \half\paren{\paren{\paren{\eb_{i}-\bar{\eb}_{i}}\cdot\eb_{i}}\bar{\eb}_{i}
	                        -\paren{\paren{\eb_{i}-\bar{\eb}_{i}}\cdot\bar{\eb}_{i}}\eb_{i}}, \nonumber \\
	                      &= -\half\paren{\eb_{i}-\bar{\eb}_{i}}, \nonumber \\
	                      &= -\wb_{i}^{*}.\label{eq8_41}	                      
\end{align}
The basis $\set{\wb_{i},\wb_{i}^{*}}$ is called a {\em Witt basis} in the theory of quadratic forms.

\section{The General Linear Group as a Spin Group}

We will now use the results in section~\ref{sec1_15} and the extension of a linear vector function to 
blades (equation~\ref{eq1_77})\footnote{
$\f{\ubar{f}}{a\W b\W\dots}=\f{\ubar{f}}{a}\W\f{\ubar{f}}{a}\W\dots$}
and the geometric algebra definition of the determinant of a linear
transformation (equation~\ref{1_79})
\footnote{$\f{\ubar{f}}{I} = \f{\det}{\ubar{f}}I$ where $I$ is the pseudoscalar.}.

We are concerned here with linear transformations on $\Re^{n,n}$ and its subspaces, especially orthogonal
transformations.  An orthogonal transformation $\ubar{R}$ is defined by the property
\be
	\paren{\ubar{R}a}\cdot\paren{\ubar{R}b} = a\cdot b
\ee
$\ubar{R}$ is called a rotation if $\f{\det}{\ubar{R}}=1$, that is, if 
\be
	\ubar{R}E_{n,n} = E_{n,n},
\ee
where $E_{n,n}=E_{n}\bar{E}_{n}^{\R}$ is the pseudoscalar for $\Re_{n,n}$ (equation~\ref{eq8_14}).  These
rotations form a group called the {\em special orthogonal group} $\SO{n}$.

From section~\ref{sec1_10} we know that every rotation can be expressed by the canonical form
\be\label{eq8_44}
	\ubar{R}a = RaR^{\R},
\ee
where $R$ is an even multivector ({\em rotor}) satisfying
\be\label{eq8_45}
	RR^{\R} = 1.
\ee
The rotors form a multiplicative group called the {\em spin group} or {\em spin representation} of $\SO{n}$,
and it is denoted by $\Spin{n}$.  $\Spin{n}$ is said to be a double covering of $\SO{n}$, since 
equation~\ref{eq8_44} shows that both $\pm R$ correspond to the same $\ubar{R}$.

From equation~\ref{eq8_45} it follows that $R^{-1}=R^{\R}$ and that the inverse of the rotation is
\be
	\ubar{R}^{\R}a= R^{\R}aR.
\ee
This implies that from the definition of the adjoint (using {\bf RR5} in appendix~\ref{RRrules})
\be\label{eq8_47}
	a\cdot\paren{\ubar{R}b} = \grade{aRbR^{\R}}{} = \grade{bR^{\R}aR}{} = b\cdot\paren{\ubar{R}^{\R}a}. 
\ee
The adjoint of a rotation is equal to its inverse.

\begin{quotation}
``It can be shown that every rotor can be expressed in exponential form
\be
	R = \pm e^{\shalf B},\mbox{ with } R^{\R} = \pm e^{-\shalf B},
\ee
where $B$ is a bivector (section~\ref{sec1_10_2}) called the generator of $R$ or $\ubar{R}$, and the minus
sign can usually be eliminated by a change in the definition of $B$.  Thus every bivector determines a unique
rotation.  The bivector generators of a spin or rotation group form a Lie algebra under the commutator product. 
{\bf This reduces the description of Lie groups to Lie algebras.}  The Lie algebra of $\SO{n}$ and 
$\Spin{n}$ is designated by $\so{n}$.  It consists of the entire bivector space $\Re_{n,n}^{2}$. Our task
will be to prove that and develop a systematic way to find them.

Lie groups are classified according to their invariants.  For the {\em classical groups} the invariants are
nondegenerate bilinear (quadratic) forms. Geometric algebra supplies us with a simpler alternative of 
invariants, namely, the multivectors which determine the bilinear forms.  As emphasized in reference~\cite{H&S},
every bilinear form can be written as $a\cdot\paren{\f{\ubar{Q}}{b}}$ where $\ubar{Q}$ is a linear operator, 
and the form is nondegenerate if $\ubar{Q}$ is nonsingular (i.e., $\f{\det}{\ubar{Q}}\ne 0$).''
\footnote{From reference \cite{DHS&VA}.} 
\end{quotation}

To prove that $e^{\frac{B}{2}}$ is a rotation if $B$ is a general bivector consider the function\footnote{
We let $a = \f{a}{0}$.}
\begin{equation}
	\f{a}{\lambda} = e^{\frac{\lambda B}{2}}ae^{-\frac{\lambda B}{2}}
\end{equation}
where $a$ is a vector. Then differentiate  $\f{a}{\lambda}$
\begin{align}
	\deriv{a}{\lambda} &= \frac{B}{2}e^{\frac{\lambda B}{2}}ae^{-\frac{\lambda B}{2}} -
	                     e^{\frac{\lambda B}{2}}ae^{-\frac{\lambda B}{2}}\frac{B}{2} \nonumber \\
	                   &= \frac{B}{2}\f{a}{\lambda} - \f{a}{\lambda}\frac{B}{2} \nonumber \\
	                   &= B\cdot \f{a}{\lambda} \\
	\derivn{a}{\lambda}{2} &= B\cdot\paren{\deriv{a}{\lambda}} = B\cdot\paren{B\cdot \f{a}{\lambda}}\\
	\left . \derivn{a}{\lambda}{r}\right |_{\lambda=0} &= 
	       \underbrace{B\cdot\paren{B\cdot\paren{\ldots\paren{B\cdot a}}\ldots}}_{r\mbox{ copies of }B}.               
\end{align}
so that the Taylor expansion of $\f{a}{1}$ is
\begin{equation}\label{eB_taylor}
	e^{\frac{B}{2}}ae^{-\frac{B}{2}} = a + B\cdot a +\frac{1}{2!}B\cdot\paren{B\cdot a} 
		+ \frac{1}{3!}B\cdot\paren{B\cdot\paren{B\cdot a}} + \cdots.
\end{equation}
Since every term in eq~(\ref{eB_taylor}) is a vector then $e^{\frac{B}{2}}ae^{-\frac{B}{2}}$ is a vector.
Finally let $a$ and $b$ be vectors then (using RR5 in appendix~\ref{RR5})
\begin{align}
	\paren{e^{\frac{B}{2}}ae^{-\frac{B}{2}}}\cdot \paren{e^{\frac{B}{2}}be^{-\frac{B}{2}}} &= 
		\grade{e^{\frac{B}{2}}ae^{-\frac{B}{2}}e^{\frac{B}{2}}be^{-\frac{B}{2}}}{} \nonumber \\
		& = \grade{e^{\frac{B}{2}}abe^{-\frac{B}{2}}}{} \nonumber \\
		& = \grade{e^{-\frac{B}{2}}e^{\frac{B}{2}}ab}{} \nonumber \\
		& = \grade{ab}{} = a \cdot b.
\end{align}
Thus $e^{\frac{B}{2}}$ is a rotor since $e^{\frac{B}{2}}ae^{-\frac{B}{2}}$ is a vector and the transformation
generated by $e^{\frac{B}{2}}$ preserves the inner product of two vectors.

For an $n$-dimensional vector space the number of linearly independent bivectors is 
$\binom{n}{2} = \frac{n!}{2!\paren{n-2}!} = \frac{n^{2}-n}{2}$, which is also the number of generators for the
spin group $\f{Spin}{n}$ which are 
$e^{\frac{\theta_{ij}\eb_{i}\W\eb_{j}}{2}} = \f{\cos}{\frac{\theta}{2}}
+\f{\sin}{\frac{\theta}{2}}\eb_{i}\W\eb_{j}\;\;\forall\;\; 1\le i < j \le n$ where $\theta_{ij}$ is the rotation
in the plane defined by $\eb_{i}$ and $\eb_{j}$.  Thus there is a one to one correspondence between the bivectors,
$B$, and the rotations in an $n$-dimensional vector space.

{\bf $\ubar{Q}$ is invariant under a rotation $\ubar{R}$ if
\be\label{eq8_49}
	\paren{\ubar{R}a}\cdot\paren{\ubar{Q}\ubar{R}b} = a\cdot\paren{\ubar{Q}b}.
\ee
Using equation~\ref{eq8_47} transforms equation~\ref{eq8_49} to
\begin{align}
	\paren{\ubar{R}a}\cdot\paren{\ubar{Q}\ubar{R}b} &= a\cdot\paren{\ubar{Q}b}, \nonumber \\
	\paren{\ubar{Q}\ubar{R}b}\cdot\paren{\ubar{R}a} &= a\cdot\paren{\ubar{Q}b}, \nonumber \\
	a\cdot \paren{\ubar{R}^{\R}\paren{\ubar{Q}\ubar{R}b}} &= a\cdot\paren{\ubar{Q}b}, \nonumber \\
	a\cdot\paren{\ubar{R}^{\R}\ubar{Q}\ubar{R}b} &= a\cdot\paren{\ubar{Q}b},\label{eq8_50}
\end{align}
or since the $a$ and $b$ in equation~\ref{eq8_50} are arbitrary
\begin{align}
	\ubar{R}^{\R}\ubar{Q}\ubar{R} &= \ubar{Q} = \ubar{R}\ubar{Q}\ubar{R}^{\R}, \\
	\ubar{R}\ubar{R}^{\R}\ubar{Q}\ubar{R} &= \ubar{R}\ubar{Q}, \\
	\ubar{Q}\ubar{R} &= \ubar{R}\ubar{Q}.
\end{align}
Thus the invariance group of consists of those rotations which commute with $\ubar{Q}$.}

This is obviously a group since if
\begin{align}
	\ubar{R}_{1}\ubar{Q} &= \ubar{Q}\ubar{R}_{1}, \\
	\ubar{R}_{2}\ubar{Q} &= \ubar{Q}\ubar{R}_{2},
\end{align}
then
\begin{align}
	\ubar{R}_{1}\ubar{R}_{2}\ubar{Q} &= \ubar{R}_{1}\ubar{Q}\ubar{R}_{2}, \\
	\ubar{R}_{1}\ubar{R}_{2}\ubar{Q} &= \ubar{Q}\ubar{R}_{1}\ubar{R}_{2}.
\end{align}

As a specific case consider the quadratic form where $\ubar{Q}b = b^{*}$.  Then 
\begin{align}
	\ubar{Q}b &= \paren{-1}^{n+1}E_{n}bE_{n}^{-1}, \\
	\ubar{Q}\ubar{R}b &= \paren{-1}^{n+1}E_{n}RbR^{\R}E_{n}^{-1}, \nonumber \\
	\ubar{R}\ubar{Q}b &= \paren{-1}^{n+1}RE_{n}bE_{n}^{-1}R^{\R}, \nonumber \\
	E_{n}RbR^{\R}E_{n}^{-1} &= RE_{n}bE_{n}^{-1}R^{\R}, \nonumber \\
	E_{n}R &= RE_{n}, \nonumber \\
	E_{n} &= RE_{n}R^{\R}.
\end{align}
Thus, the invariance of the bilinear form $a\cdot b^{*}$ is equivalent to the invariance of the 
$n$-blade $E_{n}$.

We can determine a representation for $R$ by noting the generators of any rotation in $\Re^{n,n}$ can
be written in the form
\begin{align}
	R &= e^{\frac{\theta}{2} \bm{u}\bm{v}}, \nonumber \\
      &= \bracetwo{\f{\cos}{\theta/2}}{\f{\cosh}{\theta/2}}
	    +\bracetwo{\f{\sin}{\theta/2}}{\f{\sinh}{\theta/2}}\bm{u}\bm{v}, \nonumber \\	  
	  &=\bracetwo{\f{\cos}{\theta/2}}{\f{\cosh}{\theta/2}}
	    +\bracetwo{\f{\sin}{\theta/2}}{\f{\sinh}{\theta/2}}\sum_{i<j}\paren{u^{i}\eb_{i}+\bar{u}^{i}\bar{\eb}_{i}}
	    \paren{v^{j}\eb_{j}+\bar{v}^{j}\bar{\eb}_{j}}, \label{RotorDef}
\end{align}
where $\bm{u}\cdot\bm{v}= 0$ and $\abs{\paren{\bm{u}\bm{v}}^{2}}=1$.  Equation~(\ref{RotorDef}) is what makes corresponding
the Lie algebras with the bivector commuatator algebra possible and also allows one to calculate the generators of the
corresponding Lie group.  
The generators of the most general rotation in $\Re^{n,n}$ are $\eb_{i}\eb_{j}$, $\bar{\eb}_{i}\eb_{j}$, and $\bar{\eb}_{i}\bar{\eb}_{j}$.  The question is which of these generators commute with $E_{n}$. The answer is
\begin{align}
	\eb_{i}\eb_{j}E_{n} &= E_{n}\eb_{i}\eb_{j} \label{eq8_56}\\
	\eb_{i}\bar{\eb}_{j}E_{n} &= -E_{n}\eb_{i}\bar{\eb}_{j} \label{eq8_57}\\
	\bar{\eb}_{i}\bar{\eb}_{j}E_{n} &= E_{n}\bar{\eb}_{i}\bar{\eb}_{j}.\label{eq8_58}
\end{align}
Equation~\ref{eq8_58} is obvious since $\bar{\eb}_{i}$ and $\bar{\eb}_{j}$ give the same number of
sign flips, $n$, in passing through $E_{n}$ totalling $2n$ an even number. Likewise, 
in equation~\ref{eq8_56} as $\eb_{i}$ and $\eb_{j}$ give the same number of sign flips, $n-1$, in passing through $E_{n}$ totalling $2\paren{n-1}$ an even number.  In equation~\ref{eq8_57} $\eb_{i}$ produces $n-1$ sign flips
in traversing $E_{n}$ and $\bar{\eb}_{j}$ produces $n$ sign flips for the same traverse so that the total
number of sign flips is $2n-1$ and odd number.

Thus the reqired rotation generators for $R$ are
\begin{align}
	\eb_{ij} &= \eb_{i}\eb_{j}, \mbox{ for }i<j=1,\dots,n, \\
	\bar{\eb}_{ij} &= \bar{\eb}_{i}\bar{\eb}_{j}, \mbox{ for }i<j=1,\dots,n.
\end{align}
Also note that since $\eb_{ij}\times\bar{\eb}_{kl} = 0$, the barred and unbarred generators commute. Any generator
in the algebra can be written in the form
\be
	B = \alphab\edot\eb+\betab\edot\bar{\eb},
\ee
where
\be\label{colon_sum}
	\alphab\edot\eb \equiv \sum_{i<j}\alpha^{ij}\eb_{ij},
\ee
and the $\alpha^{ij}$ are scalar coefficients. So that $\alphab\edot\eb$ is the generator for any rotation in $\Re^{n}$ and $\betab\edot\bar{\eb}$ is the generator for any rotation in $\bar{\Re}^{n}$. The corresponding
group rotor is (we can factor the exponent since $\eb$ and $\bar{\eb}$ generators commute)
\be
	R = e^{\shalf\paren{\alphab\edot\eb+\betab\edot\bar{\eb}}} = e^{\shalf \alphab\edot\eb}
	                                                             e^{\shalf\betab\edot\bar{\eb}}.
\ee
This is the spin representation of the product group $\SO{n}\otimes\SO{n}$.

In most cases the generators of the invariance group are not as obvious and in the case of the $a\cdot b^{*}$
form.  Thus we need some general methods for such determinations.  Consider a skew-symmetric bilinear form 
$\ubar{Q}$\footnote{This selection is done with malice aforethought.  To generate the memembers of $GL\paren{n,\Re}$
we do not need the most general linear transformation on $\Re^{n,n}$ since the dimension of that group is $4n^{2}$ and
not the $n^{2}$ of $\f{GL}{n,\Re}$.} 
\be
	a\cdot\paren{\ubar{Q}b} = -b\cdot\paren{\ubar{Q}a}.
\ee 
The form $\ubar{Q}$ can be written
\be\label{eq8_65}
	a\cdot\paren{\ubar{Q}b} = a\cdot\paren{b\cdot Q} = \paren{a\W b}\cdot Q,
\ee
where $Q$ is a bivector.\footnote{This is obvious from the properties of the bilinear form that if
$a\cdot\paren{\ubar{Q}b} = \paren{a\W b}\cdot Q$ then $a\cdot\paren{\ubar{Q}b}$ is linear in $a$ and $b$ and
is skew-symmetric
\begin{align*}
	b\cdot\paren{\ubar{Q}a} &= \paren{b\W a}\cdot Q \\
	                        &= -\paren{a\W b}\cdot Q \\
	                        &= -a\cdot\paren{\ubar{Q}b}
\end{align*}
Finally the maximum number of free parameters (coefficients) for the bivector, $Q$, in an $n$-dimensional space is 
$\binom{n}{2} = \frac{n\paren{n-1}}{2}$.  This is also the number of independent coefficients in the $n\times n$ antisymmetric
matrix that represents the skew-symmetric bilinear form.}  We say that the bivector $Q$ in {\em involutory} if $\ubar{Q}$ is nonsingular and
\be\label{eq8_66}
	\ubar{Q}^{2} = \pm\ubar{1}.
\ee
Note that the operator equation~\ref{eq8_66} only applies to vectors.

Note that
\begin{align}
	\paren{\ubar{R}a\W\ubar{R}b}\cdot Q &= \ubar{R}\paren{a\W b}\cdot Q \nonumber \\
                                        &= \paren{R\paren{a\W b}R^{\R}}\cdot Q \nonumber \\
                                        &= \grade{R\paren{a\W b}R^{\R}Q}{} \nonumber \\
                                        &= \grade{\paren{a\W b}R^{\R}QR}{} \nonumber \\
                                        &= \paren{a\W b}\cdot\paren{\ubar{R}^{\R}Q}.
\end{align}
Then equation~\ref{eq8_65} gives for a stability condition 
\begin{align}
	\paren{\ubar{R}a}\cdot\paren{\ubar{Q}\ubar{R}b} &= a\cdot\paren{\ubar{Q}b}, \nonumber \\
	\paren{\paren{\ubar{R}a}\cdot\paren{\ubar{R}b}}\cdot Q &= \paren{a\W b}\cdot Q, \nonumber \\
    \paren{\paren{\ubar{R}a}\W\paren{\ubar{R}b}}\cdot Q &=  \nonumber \\
    \ubar{R}\paren{a\W b}\cdot Q &= \nonumber \\
	\grade{R\paren{a\W b}R^{\R}Q}{} &= \nonumber \\
	\grade{\paren{a\W b}R^{\R}QR}{} &= \nonumber \\
	\paren{a\W b}\cdot\paren{R^{\R}QR} &= \paren{a\W b}\cdot Q, \nonumber \\		
	 R^{\R}QR &= Q.\nonumber \\
	 QR &= RQ\label{eq8_68}
\end{align}
From equation~\ref{eq8_68} we have that generators of the stability group $\f{G}{Q}$ for $Q$ must
commute with $Q$.  To learn more about this requirement, we study the commutator of $Q$ with an
arbitrary bivector blade $a\W b$.  Since $a\W b=a\times b$ and $a\times Q = a\cdot Q$ the Jacobi 
identity gives
\begin{align}
	\paren{a\W b}\times Q &= \paren{a\times b}\times Q, \nonumber \\
	                      &= \paren{a\times Q}\times b+a\times\paren{b\times Q}, \nonumber \\
	                      &= \paren{a\times Q}\W b+a\W\paren{b\times Q}, \nonumber \\
	                      &= \paren{a\cdot Q}\W b+a\W\paren{b\cdot Q}, \nonumber \\
	                      &= \paren{\ubar{Q}a}\W b+a\W\paren{\ubar{Q}b},\label{eq8_69}
\end{align}
and then
\be
	\paren{\paren{a\W b}\times Q}\times Q = \paren{\paren{\ubar{Q}a}\W b
	                                         +a\W\paren{\ubar{Q}b}}\times Q.	                                           
\ee
Now using the Jacobi identity and the extension of linear functions to blades we have
\begin{align}
	 \paren{\paren{\ubar{Q}a}\W b}\times Q &= \paren{\paren{\ubar{Q}a}\times b}\times Q, \nonumber \\
	                                       &= \paren{\ubar{Q}a}\times\paren{b\times Q}
	                                          -b\times\paren{\paren{\ubar{Q}a}\times Q}, \nonumber \\
	                                       &= \paren{\ubar{Q}a}\W\paren{b\cdot Q}
	                                          -b\W\paren{\paren{\ubar{Q}a}\cdot Q}, \nonumber \\
	                                       &= \paren{\ubar{Q}a}\W\paren{\ubar{Q}b}
	                                          -b\W\paren{\ubar{Q}^{2}a}, \nonumber \\
	                                       &= \paren{\ubar{Q}^{2}a}\W b+\f{\ubar{Q}}{a\W b}.
\end{align}
Similarly
\be
	\paren{a\W\paren{\ubar{Q}b}}\times Q = \f{\ubar{Q}}{a\W b}+a\W\paren{\ubar{Q}^{2}b},
\ee
so that, since $Q$ is involutary $\paren{\ubar{Q}^{2}=\pm\ubar{1}}$
\begin{align}
	\paren{\paren{a\W b}\times Q}\times Q &= \paren{\ubar{Q}^{2}a}\W b+2\ubar{Q}\paren{a\W b}
	                                         +a\W\paren{\ubar{Q}^{2}b}, \nonumber \\
	                                      &= 2\paren{\f{\ubar{Q}}{a\W b}\pm a\W b}.\label{eq8_73}
\end{align}
By linearity and superposition since equation~\ref{eq8_73} holds for any blade $a\W b$ it also holds
for any bivector (superposition of 2-blades) $B$ so that
\be
	\paren{B\times Q}\times Q = 2\paren{\ubar{Q}B\pm B}.
\ee
If $B$ commutes with $Q$ then $B\times Q = 0$ and
\begin{align}
	\ubar{Q}B\pm B &= 0, \\
	\ubar{Q}B &= \mp B.
\end{align}
Thus the generators of $\f{G}{Q}$ are the eigenbivectors of $\ubar{Q}$ with eigenvalues $\mp 1$.

Now define the bivectors $\f{E^{\pm}}{a,b}$ and $\f{F}{a,b}$ by
\begin{align}
	\f{E^{\pm}}{a,b} &\equiv a\W b \pm\paren{\ubar{Q}a}\paren{\ubar{Q}b}, \label{eq8_77}\\
	\f{F}{a,b} &\equiv \paren{\ubar{Q}a}\W b -a\W\paren{\ubar{Q}b}. \label{eq8_78}
\end{align}
Then using equation~\ref{eq8_69} we get
\begin{align}
	\f{E^{\pm}}{a,b}\times Q &= \paren{\ubar{Q}a}\W b+a\W\paren{\ubar{Q}b}
	                            \mp\paren{\ubar{Q}^{2}a}\W\paren{\ubar{Q}b}
	                            \mp\paren{\ubar{Q}a}\W\paren{\ubar{Q}^{2}b}, \nonumber \\
	                         &= \paren{\ubar{Q}a}\W b+a\W\paren{\ubar{Q}b}
	                            -a\W\paren{\ubar{Q}b}-\paren{\ubar{Q}a}\W b, \nonumber \\
	                         &= 0, \\
	\f{F}{a,b}\times Q &= \paren{\ubar{Q}a}\W\paren{\ubar{Q}b}+a\W\paren{\ubar{Q}^{2}b}
	                      -\paren{\ubar{Q}^{2}a}\W b-\paren{\ubar{Q}a}\W\paren{\ubar{Q}b}, \nonumber \\
	                   &= \pm a\W b \mp a\W b, \nonumber \\
	                   &= 0.      
\end{align}
Thus $\f{E^{\pm}}{a,b}$ and $\f{F}{a,b}$ are the generators of the stability qroup for $Q$.  A basis
for the Lie algebra is obtained by inserting basis vectors for $a$ and $b$.  The commutation relations
for the generators $\f{E^{\pm}}{a,b}$ and $\f{F}{a,b}$ can be found from equations~\ref{eq8_77},
\ref{eq8_78}, and \ref{eq453}.  Evaluation of the commutation relations is simplified by using the 
eigenvectors of $\ubar{Q}$ for a basis, so it is best to defer the task until $\ubar{Q}$ is completely
specified.

As a concrete example let $Q=K$ where $K$ is from equation~\ref{eq8_22} and use equations~\ref{eq8_38} and
\ref{eq8_39} to get\footnote{Since $\ubar{K}^{2}=1$ we only need consider $\f{E^{+}}{\eb_{i},\eb_{j}}$.}
\begin{align}
	E_{ij} = \f{E^{+}}{\eb_{i},\eb_{j}} &= \eb_{i}\W\eb_{j}
	                                     -\paren{\ubar{K}\eb_{i}}\W\paren{\ubar{K}\eb_{i}}, \nonumber \\
	                                   &= \eb_{i}\eb_{j}-\bar{\eb}_{i}\bar{\eb}_{j}\:\paren{i<j}, \\
    F_{ij} = \f{F}{\eb_{i},\eb_{j}}    &= \eb_{i}\W\paren{\ubar{K}\eb_{j}}
                                         -\paren{\ubar{K}\eb_{i}}\W\eb_{j}, \nonumber \\
                                       &= \eb_{i}\bar{\eb}_{j}- \bar{\eb}_{i}\eb_{j}\:\paren{i<j}, \\
    K_{i} = \half F_{ii} &= \eb_{i}\bar{\eb_{i}}.
\end{align}
At this point we should note that if a bivector $B$ is a linear combination of $E_{ij}$, $F_{ij}$, and 
$F_{i}$, $B$ will commute with $Q$. Then $R = e^{\frac{B}{2}}$ also commutes with
$Q$ since $e^{\frac{B}{2}}$ only can contain powers of $B$.  Also  $E_{ij}$, $F_{ij}$, and 
$F_{i}$ are called the generators of the Lie algebra associated with the bilinear form defined by
$\ubar{K}$ and the number of generators are $\frac{n^{2}-n}{2}+ \frac{n^{2}-n}{2}+n = n^{2}$. 

The structure equations for the Lie algebra (non-zero commutators of the Lie algebra generators) of the bilinear form
$\ubar{K}$ are
\begin{align}
	E_{ij}  \times F_{ij} &=  2\paren{K_{i}-K_{j}}\\
	E_{ij}  \times K_{i}  &= -F_{ij}\\
	F_{ij}  \times K_{i}  &= -E_{ij}\\
	E_{ij}  \times E_{il} &= -E_{jl}\\
	F_{ij}  \times F_{il} &= E_{jl}\\
	F_{ij}  \times E_{il} &= F_{jl}.
\end{align}
The structure equations close the algebra with respect to the commutator product (see appendix~\ref{sympy_glg} for how 
to calculate the structure equations).

Thus (using the notation of eq~(\ref{colon_sum})\footnote{The $\bm{\alpha : E}$ notation simply says 
that indices of the scalar coefficients, $\alpha_{ij}$, are balanced by the indices of the 
bivectors, $E_{ij}$.  So that $\bm{\alpha : E} = \sum_{i<j}\alpha_{ij}E_{ij}$ or 
$\bm{\gamma : K} = \sum_{i}\gamma_{i}K_{i}$.}) the rotors for the stablity group of $\ubar{K}$ can be
written
\begin{equation}
	e^{\frac{B}{2}} = e^{\frac{1}{2}\paren{\bm{\alpha : E} + \bm{\beta : F} + \bm{\gamma : K}}},
\end{equation}
where the $n^{2}$ coefficients are $\alpha_{ij}$, $\beta_{ij}$, and $\gamma_{i}$.

The stability group of $K$ can be identified with the {\em general linear group} $\GL{n,\Re}$. First we
must show that $\ubar{K}$ does not mix the subspaces $\mathcal{V}^{n}$ and $\mathcal{V}^{n*}$ of
$\Re^{n,n}$.  Using equations~\ref{eq8_40} and \ref{eq8_41} we have (where $W_{n}$ and $W_{n}^{*}$ are the
pseudoscalars for $\mathcal{V}^{n}$ and $\mathcal{V}^{n*}$).
\begin{align}
	W_{n} &= \wb_{1}\dots\wb_{n}, \\
	W_{n}^{*} &= \wb_{1}^{*}\dots\wb_{n}^{*} \\
	\f{\ubar{K}}{W_{n}} &= W_{n}, \\
	\f{\ubar{K}}{W_{n}^{*}} &= \paren{-1}^{n}W_{n}^{*}.	
\end{align}
Thus when restricted to $\mathcal{V}^{n}$ or $\mathcal{V}^{n*}$, $\ubar{K}$ is non-singular since the
$\f{\det}{\ubar{K}}\ne 0$.

Since each group element leaves $\mathcal{V}^{n}$ invariant 
\begin{equation}
\ubar{R}\f{\ubar{K}}{W_{n}} = \ubar{R}W_{n} = RW_{n}R^{\R} = W_{n},	
\end{equation}
we can write
\begin{equation}
	\ubar{R}\bm{w}_{j} = \sum_{k=1}^{n}\bm{w}_{k}\rho_{kj}.
\end{equation}
Where using eq~(\ref{eq8_4}) we get
\begin{equation}
	\rho_{ij} = 2\bm{w^{*}}_{i}\cdot\paren{\ubar{R}\bm{w}_{j}} = 2\grade{\bm{w^{*}}_{i}R\bm{w}_{j}R^{\R}}{}.
\end{equation}

The rotations can be described on $\mathcal{V}^{n}$ without reference to $\Re^{n}$. For any member $\ubar{R}$
of $\f{GL}{n,\Re}$ we have
\begin{equation}
	\ubar{R}W_{n} = W_{n}\f{{\det}_{\mathcal{V}^{n}}}{\ubar{R}},
\end{equation}
where ${\det}_{\mathcal{V}^{n}}$ is the determinant of the linear transformation restricted to the $\mathcal{V}^{n}$
subspace of $\Re^{n,n}$ and not on the entire vector space.  First note that
\begin{equation}
	W_{n}^{*}\cdot W_{n}^{\R} = \paren{w_{1}^{*}\ldots w_{n}^{*}}\cdot\paren{w_{n}\ldots w_{1}} =
	                           \grade{w_{1}^{*}\ldots w_{n}^{*}w_{n}\ldots w_{1}}{} = 2^{-n}.
\end{equation}
Therefore,
\begin{align}
	\ubar{R}W_{n}^{\R} &= W_{n}^{\R}\f{{\det}_{\mathcal{V}^{n}}}{\ubar{R}^{-1}}
\end{align}

\section{Endomorphisms of $\Re^{n}$}

For a vector space an \emph{endomorphism} is a linear mapping of the vector space onto itself. If the \emph{endomorphism} has an
inverse it is an \emph{automorphism}. By studying the \emph{endomorphisms} of $\Re_{n}$, the geometric algebra of $\Re^{n}$ we can
also show in an alternative way that the mother algebra, $\Re_{n,n}$, is the appropriate arena for the study of linear transformations
and Lie groups.  First note that ($\simeq$ is the symbol for ``is isomorphic to'')
\begin{equation}
	\f{\End}{\Re_{n}} \simeq \Re^{2^{2n}},
\end{equation}
since $\f{End}{\Re_{n}}$ is isomorphic to the algebra of all $2^{n}\times 2^{n}$ matrices. 

For an arbitrary multivector $A\in \Re_{n}$, left and right multiplications by orthonormal basis vectors $\eb_{i}$ determine
endomorphisms of $\Re_{n}$ defined by
\begin{align}
	\ubar{\eb}_{i}: A \rightarrow \f{\ubar{\eb}_{i}}{A} &\equiv \eb_{i}A, \\
	\bar{\ubar{\eb}}_{i}: A \rightarrow \f{\bar{\ubar{\eb}}_{i}}{A} &\equiv \bar{A}\eb_{i}
\end{align}
and the involution operator $\paren{\bar{\bar{A}}=A}$ is defined by\footnote{Since any grade $r$ basis blade is of the form $\eb_{j_{1}}\ldots \eb_{j_{r}}$
with the $j_{k}$'s in normal order then
\begin{equation*}
	\overline{\eb_{j_{1}}\ldots \eb_{j_{r}}} = \bar{\eb}_{j_{1}}\ldots \bar{\eb}_{j_{r}}  = \paren{-1}^{r}\eb_{j_{1}}\ldots \eb_{j_{r}}
\end{equation*}
and the involute of any multivector, $A$, can be determined from equations~(\ref{inv_eq3}) and (\ref{inv_eq4}).
Likewise $\overline{\overline{\eb_{j_{1}}\ldots \eb_{j_{r}}}}=\eb_{j_{1}}\ldots \eb_{j_{r}}$.}
\begin{align}
	\overline{\paren{AB}} &= \bar{A}\bar{B} \label{inv_eq3}\\
	\bar{\eb}_{i} &= -\eb_{i}.\label{inv_eq4}
\end{align}
Thus for any multivector $A$ we have
\begin{align}
	\f{\ubar{\eb}_{i}\ubar{\eb}_{j}}{A} &= \f{\ubar{\eb}_{i}}{\eb_{j}A} = \eb_{i}\eb_{j}A, \\
	\f{\ubar{\eb}_{i}\ubar{\bar{\eb}}_{j}}{A} &= \f{\ubar{\eb}_{i}}{\bar{A}\eb_{j}} = \eb_{i}\bar{A}\eb_{j}, \\
	\f{\ubar{\bar{\eb}}_{j}\ubar{\eb}_{i}}{A} &= \f{\ubar{\bar{\eb}}_{j}}{\eb_{i}A} = \overline{\eb_{i}A}\eb_{j} = -\eb_{i}\bar{A}\eb_{j}, \\
	\f{\ubar{\bar{\eb}}_{i}\ubar{\bar{\eb}}_{j}}{A} &= \f{\ubar{\bar{\eb}}_{i}}{\bar{A}\eb_{j}} = \overline{\bar{A}\eb_{j}}\eb_{i} =
	                                                   -A\eb_{j}\eb_{i}.
\end{align}
Thus
\begin{align}
	\f{\paren{\ubar{\eb}_{i}\ubar{\eb}_{j}+\ubar{\eb}_{j}\ubar{\eb}_{i}}}{A} &= \paren{\eb_{i}\eb_{j}+\eb_{j}\eb_{i}}A = 2\delta_{ij}A, \nonumber \\
	\ubar{\eb}_{i}\ubar{\eb}_{j}+\ubar{\eb}_{j}\ubar{\eb}_{i} &= 2\delta_{ij}, \label{Eeiej}\\
	\f{\paren{\ubar{\eb}_{i}\ubar{\bar{\eb}}_{j}+\ubar{\bar{\eb}}_{j}\ubar{\eb}_{i}}}{A} &= \eb_{i}A\eb_{j}-\eb_{i}A\eb_{j} = 0, \nonumber \\
	\ubar{\eb}_{i}\ubar{\bar{\eb}}_{j}+\ubar{\bar{\eb}}_{j}\ubar{\eb}_{i} &= 0, \label{Eeibej}\\	
	\f{\paren{\ubar{\bar{\eb}}_{i}\ubar{\bar{\eb}}_{j}+\ubar{\bar{\eb}}_{j}\ubar{\bar{\eb}}_{i}}}{A} &= 
	                              -A\paren{\eb_{j}\eb_{i}+\eb_{i}\eb_{j}}A = -2\delta_{ij}A, \nonumber \\
	\ubar{\eb}_{i}\ubar{\eb}_{j}+\ubar{\eb}_{j}\ubar{\eb}_{i} &= -2\delta_{ij}.\label{Ebeibej}
\end{align}
Thus equations~(\ref{Eeiej}), (\ref{Eeibej}), and (\ref{Ebeibej}) are isomorphic to equations~(\ref{eiej}), (\ref{eibej}), and 
(\ref{beibej}) which define the orthogonal basis $\set{\eb_{i},\bar{\eb}_{j}}$ of $\Re^{n,n}$.  This establishes the isomorphism\footnote{
$\f{\dim}{\Re_{n,n}}=2^{2n}$.}
\begin{equation}
	\Re_{n,n} \simeq \f{\End}{\Re_{n}}.
\end{equation}
The differences between $\Re_{n,n}$ and $\f{\End}{\Re_{n}}$ are that $\Re_{n,n}$ is the geometric algebra of a vector space with signature
$\paren{n,n}$, basis $\set{\eb_{i},\bar{\eb}_{j}}$, and dimension $2^{2n}$, while $\f{\End}{\Re_{n}}$ are the linear mappings of the 
geometric algebra $\Re_{n}$ on to itself with linear basis functions $\set{\ubar{\eb}_{i},\ubar{\bar{\eb}}_{j}}$ that are isomorphic to 
$\set{\eb_{i},\bar{\eb}_{j}}$ and have the same anti-commutation relations (dot products) as $\set{\eb_{i},\bar{\eb}_{j}}$.

Note that in defining $\f{\ubar{\bar{\eb}}_{i}}{A} = \bar{A}\eb_{i}$, the involution, $\bar{A}$, is required to get the proper 
anti-commutation (dot product) relations that insure $\ubar{\eb}_{i}$ and $\ubar{\bar{\eb}}_{j}$ are orthogonal and that the 
signature of $\f{\End}{\Re_{n}}$ is $\paren{n,n}$.

Additionally, the composite operators
\begin{align}
	\ubar{\eb}_{i}\ubar{\bar{\eb}}_{i}&:A\rightarrow \f{\ubar{\eb}_{i}\ubar{\bar{\eb}}_{i}}{A} = \eb_{i}\bar{A}\eb_{i} \\
	\f{\ubar{\eb}_{i}\ubar{\bar{\eb}}_{i}}{AB} &=  \eb_{i}\bar{A}\bar{B}\eb_{i} \nonumber \\
	                                           &=  \eb_{i}\bar{A}\eb_{i}\eb_{i}\bar{B}\eb_{i} \nonumber \\
	                                           &= \f{\ubar{\eb}_{i}\ubar{\bar{\eb}}_{i}}{A}\f{\ubar{\eb}_{i}\ubar{\bar{\eb}}_{i}}{B}
\end{align}
preserve the geometric product and generate $\f{\Aut}{\Re_{n}}$, a subgroup of $\f{\End}{\Re_{n}}$.



