\chapter{Classical Electromagnetic Theory}
To formulate classical electromagnetic theory in terms of geometric algebra we will use the space-time algebra with orthogonal basis vectors $\g{0}$,
$\g{1}$, $\g{2}$, and $\g{3}$ and indexing convention that latin indices take on values 1 through 3 and greek indices values 0 through 3. The
signature of the vector space is given by $\g{0}^{2} = -\g{i}^{2} = 1$ and the reciprocal basis is given by $\gr{0} = \g{0}$ and $\gr{i} = -\g{i}$. The
relative basis vectors are given by $\Sig{i} = \g{i}\g{0}$. These bivectors are called relative vectors because their multiplication table is the same
as for the geometric algebra of Euclidean 3-space (section~\ref{subsect_Euclidian}) since $\Sig{i}^{2} = 1$ and $\Sig{i}\Sig{j}=-\Sig{j}\Sig{i}$.  
Vector accents are used for relative vectors since it is usually used to denote vectors in ordinary (Euclidean) 3-space. We also have 
that $\Sig{1}\Sig{2}\Sig{3} = \g{0}\g{1}\g{2}\g{3} = I$.

We now define relative electric and magnetic field vectors
\begin{align}
	\Ev &\equiv E\g{0} = E^{i}\g{i}\g{0} = E^{i}\Sig{i} \\
	\Bv &\equiv B\g{0} = B^{i}\g{i}\g{0} = B^{i}\Sig{i}
\end{align}
We also have
\begin{align}
	\nabla\g{0} &= \paren{\gr{0}\partial_{0}+\gr{i}\partial_{i}}\g{0} \\
	            &= \partial_{0}-\Sig{i}\partial_{i} \\
	            &= \partial_{0}-\Nabla
\end{align}
where we have defined
\be
	\Nabla \equiv \Sig{i}\partial_{i}
\ee
and also have
\be
	\g{0}\nabla = \partial_{0}+\g{0}\gr{i}\partial_{i} = \partial_{0}-\g{0}\g{i}\partial_{i} = \partial_{0}+\g{i}\g{0}\partial_{i} = \partial_{0}+\Nabla
\ee
Finally we have the 4-current $J = J^{\mu}\g{\mu}$ where $J^{0}=\rho$ then
\be
	J\g{0} = \rho+J^{i}\g{i}\g{0} = \rho+J^{i}\Sig{i} = \rho+\Jv.
\ee
\section{Maxwell Equations}
The four vacuum Maxwell equations are then given by (for this section $\times$ is the 3-D vector product and remember in three dimensions $a\times b=-I(a\W b)$)
\begin{align}
	\Nabla\cdot\Ev &= \rho \label{eq8_8}\\
	\Nabla\cdot\Bv &= 0  \label{eq8_9}\\
	-\Nabla\times\Ev &= I\Nabla\W\Ev = \partial_{0}\Bv \label{eq8_10}\\
	\Nabla\times\Bv &= -I\Nabla\W\Bv = \partial_{0}\Ev+\Jv \label{eq8_11}
\end{align}
Now multiply equations~\ref{eq8_9} and \ref{eq8_10} by $I$ and $-I$ respectively and add \ref{eq8_10} to \ref{eq8_11} and \ref{eq8_8} to \ref{eq8_9} to get
\begin{align}
	\Nabla\cdot\paren{\Ev+I\Bv} &= \rho \\
	\Nabla\W\paren{\Ev+I\Bv} &= -\partial_{0}\paren{\Ev+I\Bv}-\Jv \\
	\Nabla\W\paren{\Ev+I\Bv}+\partial_{0}\paren{\Ev+I\Bv} &= -\Jv
\end{align}
Let
\be
	F = \Ev+I\Bv
\ee
so that
\begin{align}
	\Nabla\cdot F = \rho \\
	\Nabla\W F+\partial_{0}F = -\Jv.
\end{align}
Now remember ($\g{0}=\gr{0}$)
\begin{align}
 	\Nabla F &= \Nabla\W F+\Nabla\cdot F \nonumber \\
 	\Nabla F &= -\partial_{0}F+\rho-\Jv \nonumber \\
 	\Nabla F + \partial_{0}F &= \rho-\Jv  \nonumber \\
 	\g{0}\nabla F &= \rho-\Jv \nonumber \\
 	\nabla F &= \rho\g{0}-J^{i}\g{0}\g{i}\g{0} \nonumber \\
	\nabla F &= \rho\g{0}+J^{i}\g{i}
\end{align}
or
\color{red}
\be
	\fbox{$\nabla F = J.  \label{eq8_18a}$}
\ee
\normalcolor

Equation~\ref{eq8_18a} contains all four vector Maxwell equations.  Evaluating $F$ gives
\be
	F = E^{1}\g{0}\g{1}+E^{2}\g{0}\g{2}+E^{3}\g{0}\g{3}+B^{1}\g{2}\g{3}-B^{2}\g{1}\g{3}+B^{3}\g{1}\g{2}
\ee
with the conserved quatities $\Ev^{2}-\Bv^{2}$ and $\Ev\cdot\Bv$ since
\be
	F^{2} = \Ev^{2}-\Bv^{2}+2\paren{\Ev\cdot\Bv}I = -FF^{\R}
\ee
and both scalars and pseudoscalars are unchange by spacetime rotations (Lorentz transformations).

The components of $F$ as a tensor are given by $F^{\mu\nu} =\paren{\gr{\nu}\W\gr{\mu}}\cdot F$
\begin{equation}
F^{\mu\nu} = \left [ 
\begin{array}{cccc}  
0 & - {E^{1}} & - {E^{2}} & - {E^{3}} \\ 
{E^{1}} & 0 & - {B^{3}} & {B^{2}} \\ 
{E^{2}} & {B^{3}} & 0 & - {B^{1}} \\ 
{E^{3}} & - {B^{2}} & {B^{1}} & 0
\end{array} 
\right ] 
\end{equation}
\section{Relativity and Particles}\label{sect8_2}
Let $\f{x}{\lambda} = \f{x^{\nu}}{\lambda}\gamma_{\nu}$ be a parameterization of a particles world line (4-vector).  
The the proper time, $\tau$, of the world line is the parameterization $\f{\lambda}{\tau}$ such that
\be
	\paren{\deriv{x}{\tau}}^{2} = 1.
\ee
Since we are only considering physically accessible world lines where a particle velocity cannot equal or exceed the 
speed of light we have
\be\label{eq8_24}
	\paren{\deriv{x}{\lambda}}^{2} > 0.
\ee
The function $\f{\lambda}{\tau}$ can always be constructed as follows -
\begin{align}
	\deriv{x}{\tau} &= \deriv{x}{\lambda}\deriv{\lambda}{\tau} \\
	\paren{\deriv{x}{\tau}}^{2} &= \paren{\deriv{x}{\lambda}}^{2}\paren{\deriv{\lambda}{\tau}}^{2} = 1 \\
	\deriv{\lambda}{\tau} &= \bfrac{1}{\sqrt{\paren{\deriv{x}{\lambda}}^{2}}} \\
	\f{\tau}{\lambda} &= \int_{\lambda_{0}}^{\lambda}\sqrt{\paren{\deriv{x}{\lambda'}}^{2}} d\lambda'. \label{eq8_28a}
\end{align}
From the form of equation~\ref{eq8_28a} and equation~\ref{eq8_24} we know that $\f{\tau}{\lambda}$ is a monotonic function 
$\paren{\f{\tau}{\lambda_{2}}>\f{\tau}{\lambda_{1}}\;\forall\;\lambda_{2}>\lambda_{1}}$ so that the inverse function
$\f{\lambda}{\tau}$ exists $\f{x}{\tau} = \f{x}{\f{\lambda}{\tau}}$.

Now define the 4-velocity $v$ of a particle with world line $\f{x}{\tau}$ as
\be
	v \equiv \deriv{x}{\tau} = \deriv{x^{\nu}}{\tau}\gamma_{\nu} = v^{\nu}\gamma_{\nu},
\ee
the relative 3-velocity (bivector) $\vec{\beta}$ as (remember that $x^{0}$ is the time coordinate in the local coordinate frame
and $\tau$ is the time coordinate in the rest frame of the particle defined by $\vec{\beta}=0$)
\be
	\vec{\beta} \equiv \deriv{x^{i}}{x^{0}}\gamma_{i}\gamma_{0} = \deriv{x^{i}}{x^{0}}\Sig{i},
\ee
and the relativistic $\gamma$ factor as
\be
	\gamma \equiv \deriv{x^{0}}{\tau}.
\ee
Thus we can write
\begin{align}
	v &= \deriv{x^{0}}{\tau}\paren{\gamma_{0}+\deriv{x^{i}}{x^{0}}\gamma_{i}} \\
	v &= \gamma\paren{\gamma_{0}+\vec{\beta}\gamma_{0}} = \gamma\paren{1+\vec{\beta}}\gamma_{0} \\
	v^{2} &= \gamma^{2}\paren{1-\vec{\beta}^{2}} = 1 \\
	\gamma^{2} &= \bfrac{1}{1-\vec{\beta}^{2}} \\
	\gamma &= \bfrac{1}{\sqrt{1-\vec{\beta}^{2}}}.
\end{align}
We also define the relativistic acceleration $\dot{v}$ by
\be
	\dot{v} \equiv \deriv{v}{\tau}.
\ee
Note that since $v^{2}=1$ we have
\be
	\deriv{}{\tau}\paren{v\cdot v} = 2v\cdot\deriv{v}{\tau} = 2v\cdot\dot{v} = 0.
\ee


\section{Lorentz Force Law}
From section~\ref{sect8_2} we have for the relativistic 4-velocity
\begin{equation}
\deriv{x}{\tau} = \gamma\paren{1+\vec{\beta}}\gamma_{0}
\end{equation}
Then the covariant Lorentz force law is given by ($q$ is the charge of the particle)
\begin{align}
\deriv{p}{\tau} = q\deriv{x}{\tau}\cdot F &= q\gamma\lp\lp {E^{1}} {\beta^{1}} + {E^{1}} {\beta^{1}} + {E^{3}} {\beta^{3}}\rp {\gamma}_{0} \right . \nonumber \\
                & \hspace{12pt}+ \lp - {B^{2}} {\beta^{3}} + {B^{3}} {\beta^{2}} + {E^{1}}\rp {\gamma}_{1} \nonumber \\
                & \hspace{12pt}+ \lp {B^{1}} {\beta^{3}} - {B^{3}} {\beta^{1}} + {E^{2}}\rp {\gamma}_{2}  \nonumber \\
                & \hspace{12pt}+ \left . \lp - {B^{1}} {\beta^{2}} + {B^{2}} {\beta^{1}} + {E^{3}}\rp {\gamma}_{3} \rp  \\
                &=  q\gamma\paren{\paren{\vec{E}\cdot\vec{\beta}}\gamma_{0}+\paren{\paren{\vec{E}+\vec{\beta}\times\vec{B}}\cdot\Sig{i}}\gamma_{i}}
\end{align}
where the component $q\gamma\paren{\vec{E}\cdot\vec{\beta}}$ is the time derivative of the work done on the 
particle by the electric field in the specified coordinate frame.
\section{Relativistic Field Transformations}
A Lorentz transformation is described by a rotor $L = BR$ which is composed of a pure spatial rotation $R$ and a relativistic boost rotation $B$.  The
general form of $B$ is 
\be
	B = \f{\cosh}{\bfrac{\alpha}{2}}-\f{\sinh}{\bfrac{\alpha}{2}}\hat{\beta}
\ee 
where $\beta$ is the magnitude of the relative 3-velocity (bivector) and $\hat{\beta}$ is a unit relative vector (bivector) in the direction of the 3-velocity.  Then
\begin{align}
	\beta &= \f{\tanh}{\alpha} \label{eq8_26}\\
	\f{\cosh}{\alpha} &= \gamma = \frac{1}{\sqrt{1-\beta^{2}}} \label{eq8_27}\\
	\f{\sinh}{\alpha} &= \gamma\beta. \label{eq8_28}
\end{align}
If a boost is performed the relationship between the old basis vectors, $\gamma_{\nu}$, and the new basis vectors, $\grave{\gamma}_{\nu}$, is
\be
	\grave{\gamma}_{\nu} = B\gamma_{\nu}B^{\R}.
\ee
If a boost is performed on the basis vectors we must have for the electomagnetic field bivector (remember $BB^{\R}=B^{\R}B=1$)
\begin{align}
	F &= \grave{F}^{\mu\nu}\grave{\gamma}_{\mu}\grave{\gamma}_{\nu} = F^{\mu\nu}\gamma_{\mu}\gamma_{\nu} \\
	F &= \grave{F}^{\mu\nu}B\gamma_{\mu}B^{\R}B\gamma_{\nu}B^{\R} = \grave{F}^{\mu\nu}B\gamma_{\mu}\gamma_{\nu}B^{\R} = F^{\mu\nu}\gamma_{\mu}\gamma_{\nu} \\
	B^{\R}FB &= \grave{F}^{\mu\nu}\gamma_{\mu}\gamma_{\nu} = F^{\mu\nu}B^{\R}\gamma_{\mu}\gamma_{\nu}B.
\end{align}
As an example consider the case of $\hat{\beta}=\g{1}\g{0}$ and
\be
	B = \f{\cosh}{\bfrac{\alpha}{2}}-\f{\sinh}{\bfrac{\alpha}{2}}\g{1}\g{0}
\ee 
where the velocity boost is along the 1-axis (x-axis). Then (after applying hyperbolic trig identities and double angle formulas)
\begin{align}
B^{\R}FB =  & \hspace{4pt}  {E^{1}}\gamma_{0}\gamma_{1} \nn \\
            & \hspace{4pt} + \left( - {B^{3}} \operatorname{sinh}\left(\alpha\right) + {E^{2}} \operatorname{cosh}\left(\alpha\right)\right)\gamma_{0}\gamma_{2} \nn \\ 
            & \hspace{4pt} + \left( {B^{3}} \operatorname{cosh}\left(\alpha\right) - {E^{2}} \operatorname{sinh}\left(\alpha\right)\right)\gamma_{1}\gamma_{2} \nn \\ 
            & \hspace{4pt} + \left( {B^{2}} \operatorname{sinh}\left(\alpha\right) + {E^{3}} \operatorname{cosh}\left(\alpha\right)\right)\gamma_{0}\gamma_{3} \nn \\ 
            & \hspace{4pt} + \left( - {B^{2}} \operatorname{cosh}\left(\alpha\right) - {E^{3}} \operatorname{sinh}\left(\alpha\right)\right)\gamma_{1}\gamma_{3} \nn\\ 
            & \hspace{4pt} + {B^{1}}\gamma_{2}\gamma_{3}.
\end{align}
Now using equations~\ref{eq8_26}, \ref{eq8_27}, and \ref{eq8_28} to get 
\begin{align}
B^{\R}FB =  & \hspace{4pt}  {E^{1}}\gamma_{0}\gamma_{1} \nn \\ 
            & \hspace{4pt} + \gamma \left(- \beta {B^{3}} + {E^{2}}\right)\gamma_{0}\gamma_{2} \nn \\ 
            & \hspace{4pt} + \gamma \left(- \beta {E^{2}} + {B^{3}}\right)\gamma_{1}\gamma_{2} \nn \\ 
            & \hspace{4pt} + \gamma \left(\beta {B^{2}} + {E^{3}}\right)\gamma_{0}\gamma_{3} \nn \\ 
            & \hspace{4pt} + \gamma \left(- \beta {E^{3}} - {B^{2}}\right)\gamma_{1}\gamma_{3} \nn \\ 
            & \hspace{4pt} + {B^{1}}\gamma_{2}\gamma_{3}
\end{align}
Equating components of $B^{\R}FB$ gives
\be
\begin{array}{cc}
\grave{E}^{1} = E^{1} & \grave{B}^{1} = B^{1} \\
\grave{E}^{2} = \gamma\paren{E^{2}-\beta B^{3}} & \grave{B}^{2} = \gamma\paren{B^{2}+\beta E^{3}} \\ 
\grave{E}^{3} = \gamma\paren{E^{3}+\beta B^{2}} & \grave{B}^{3} = \gamma\paren{B^{3}-\beta E^{2}}
\end{array}
\ee
for the transformed electomagnetic field components.
\section{The Vector Potential}
Starting with
\be
	\nabla F = \nabla\cdot F + \nabla\W F = J
\ee
and noting that $\nabla\cdot F$ and $J$ are vectors and $\nabla\W F$ is a trivector we have
\begin{align}
	\nabla\cdot F &= J	\\
	\nabla\W F    &= 0.
\end{align}
By equation~\ref{eq3_13} we can then write
\begin{align}
	F &= \nabla\W A \\
	\nabla\W F &= \nabla\W\paren{\nabla\W A} = 0.	
\end{align}
The equation $\nabla\cdot F = J$ gives (use the fact that $\nabla G = \nabla\W G+\nabla\cdot G$ where $G$ is a multivector field)
\begin{align}
	J &= \nabla\cdot\paren{\nabla\W A} \nn \\
	  &= \nabla\paren{\nabla\W A}-\nabla\W\paren{\nabla\W A} \nn \\
	  &= \nabla\paren{\nabla\W A} \nn \\
	  &= \nabla\paren{\nabla A - \nabla\cdot A} \nn \\
	  &= \nabla^{2}A-\nabla\paren{\nabla\cdot A}.
\end{align}
We also note that if $\lambda$ is a scalar field then
\be
	F = \nabla\W\paren{A+\nabla\lambda} = \nabla\W A
\ee
since $\nabla\W\paren{\nabla\lambda} = 0$.  We have the freedom to add $\nabla\lambda$ to $A$ without changing
$F$.  This is gauge invariance.

The Lorentz gauge is choosing $\nabla\cdot A= 0$, which is equivalent to $\nabla\cdot\paren{\nabla\lambda} = -\nabla\cdot A$, so that the equation for the vector potential becomes
\be
	\nabla^{2}A = J.
\ee  
pdfl\section{Radiation from a Charged Particle}
We now will calculate the electromagnetic fields radiated by a moving point charge.  Let $\f{x_{p}}{\tau}$ be the 4-vector (space-time) trajectory of the point
charge as a function of the charge's proper time, $\tau$, and let $x$ be the 4-vector observation point at which the fields are to be calculated.  The 4-vector
separation between the observer and the radiator is
\be
	X = x - \f{x_{p}}{\tau} = \paren{x^{0}-\f{x_{p}^{0}}{\tau}}\g{0}+\paren{\vec{r}-\vec{r}_{p}}\g{0}.
\ee
The critical relationship involving $X$ is that it must lie on the light cone of $\f{x_{p}}{\tau}$ since we are dealing with electromagnetic radiation.  That is
$X$ must be a null vector, $X^{2}=0$.  $\f{X}{x,\tau}^{2}=0$ implies there is a functional relationship between $\tau$ and $x$ so that we may write $\f{\tau_{r}}{x}$.  
$\f{\tau_{r}}{x}$ is the retarded time function.  We use retarded time to mean the smaller solution (earlier time) of $\f{X}{x,\tau}^{2}=0$ since the equation must always have two solutions. The larger solution would be the advanced time and would correspond to signal travelling faster than light.  In all these calculations $c=1$ and the relative 3-velocity (bivector) of the moving charge is $\vec{\beta}_{p}$
(section~\ref{sect8_2}).

The 4-velocity and acceleration of the point charge are given by
\begin{align}
	v_{p} &= \deriv{x_{p}}{\tau} = \gamma_{p}\paren{1+\vec{\beta}_{p}}\gamma_{0}\\
	\dot{v}_{p} &= \deriv{v_{p}}{\tau}
\end{align}
The vector potential of a moving point charge of charge $q$ is given by the Li\`{e}nard-Wiechart potential\footnote{en.wikipedia.org/wiki/Li\`{e}nard-Wiechert\_potential}
\begin{align}
	\f{\varphi}{\vec{r},t} &= \bfrac{q}{4\pi}\paren{\bfrac{1}{\paren{1-\vec{n}\cdot\vec{\beta_{p}}}\abs{\vec{r}-\vec{r_{s}}}}}_{\f{\tau_{r}}{x}} \\
	\f{\vec{A}}{\vec{r},t} &= \f{\vec{\beta}_{p}}{\f{\tau_{r}}{x}}\f{\varphi}{\vec{r},t}
\end{align}
where
\be
	\vec{n} = \bfrac{\vec{r}-\vec{r_{p}}}{\abs{\vec{r}-\vec{r_{p}}}} 
\ee
or in terms of 4-vectors
\be
	A = \paren{\f{\varphi}{\vec{r},t}+\f{\vec{A}}{\vec{r},t}}\gamma_{0} = \bfrac{q}{4\pi}\bfrac{v_{p}}{\abs{X\cdot v_{p}}}
\ee
and we must calculate $\nabla\W A$. In the following derivation whenever we write $X$ we mean $\f{X}{x,\f{\tau_{r}}{x}}$.  $X$ is always evaluated
at the retarded time.

To start consider that if we know $\f{\tau_{r}}{x}$ implicitly defined by $\f{X}{x,\tau_{r}}^{2} = 0$ then the function $\f{X}{x,\f{\tau_{r}}{x}}^{2} \equiv 0$ (identically
equal to zero). Thus $\nabla\paren{\f{X}{x,\f{\tau_{r}}{x}}^{2}} \equiv 0$. First note that (by the symmetry of $g_{\mu\nu}$)
\begin{align}
	\nabla\paren{X^{2}} &= \gamma^{\eta}\partial_{\eta}\paren{X^{\mu}X^{\nu}g_{\mu\nu}} \nn \\
	                    &= \gamma^{\eta}\paren{\paren{\partial_{\eta}X^{\mu}}X^{\nu}+X^{\mu}\partial_{\eta}X^{\nu}}g_{\mu\nu} \nn \\
	                    &= 2\gamma^{\eta}\paren{\partial_{\eta}X^{\mu}}X^{\nu}g_{\mu\nu} \nn \\
	                    &= 2\gamma^{\eta}\paren{\partial_{\eta}X}\cdot X. 
\end{align}
Thus $\gamma^{\eta}\paren{\partial_{\eta}X}\cdot X = 0$ is equivalent to $\nabla\paren{X^{2}}=0$. But
\begin{align}
	\gamma^{\eta}\paren{\partial_{\eta}X}\cdot X &= \gamma^{\eta}\paren{\partial_{\eta}x}\cdot X -\gamma^{\eta}\paren{\partial_{\eta}x_{p}}\cdot X  \nn \\
		                                         &= \gamma^{\eta}\paren{\pdiff{x^{\mu}}{x^{\eta}}\gamma_{\mu}}\cdot X^{\nu}\gamma_{\nu}
		                                           -\gamma^{\eta}\paren{\pdiff{x_{p}^{\mu}}{x^{\eta}}\gamma_{\mu}}\cdot X \nn \\
		                                         &= \gamma^{\eta}\delta_{\eta}^{\mu}X^{\nu}g_{\mu\nu}
		                                           -\gamma^{\eta}\paren{\pdiff{x_{p}^{\mu}}{\tau_{r}}\pdiff{\tau_{r}}{x^{\eta}}\gamma_{\mu}}\cdot X	\nn \\
		                                         &= g_{\mu\nu}\gamma^{\mu}X^{\nu}
		                                           -\gamma^{\eta}\pdiff{\tau_{r}}{x^{\eta}}\paren{\pdiff{x_{p}^{\mu}}{\tau_{r}}\gamma_{\mu}}\cdot X	\nn \\
		                                         &= X - \nabla\tau_{r}\paren{v_{p}\cdot X} = 0		                                           	                                           
\end{align}
or
\be
	\nabla\tau_{r} = \bfrac{X}{X\cdot v_{p}}.
\ee
Another simple relation we need is
\begin{align}
	\nabla v_{p} &= \gamma^{\mu}\pdiff{v_{p}^{\nu}}{x^{\mu}}\gamma_{\nu} \nn \\
	         &= \gamma^{\mu}\pdiff{v_{p}^{\nu}}{\tau_{r}}\pdiff{\tau_{r}}{x^{\mu}}\gamma_{\nu} \nn \\
	         &= \paren{\nabla\tau_{r}}\dot{v}_{p} \nn \\
	         &= \bfrac{X\dot{v}_{p}}{X\cdot v_{p}}.
\end{align}
The last quantity we need is $\nabla\paren{X\cdot v}$
\begin{align}
	\nabla\paren{X\cdot v} &= \gamma^{\mu}\partial_{\mu}\paren{\paren{x^{\nu}-x_{p}^{\nu}}v_{p}^{\eta}g_{\nu\eta}} \nn \\
	                       &= \gamma^{\mu}\paren{\paren{\pdiff{x^{\nu}}{x^{\mu}}-\pdiff{x_{p}^{\nu}}{\tau_{r}}\pdiff{\tau_{r}}{x^{\mu}}}v_{p}^{\eta}g_{\nu\eta}
	                          +X^{\nu}\pdiff{v_{p}^{\eta}}{\tau_{r}}\pdiff{\tau_{r}}{x^{\mu}}g_{\mu\nu}} \nn \\
	                       &= \gamma^{\nu}g_{\nu\eta}v_{p}^{\eta}-\gamma^{\mu}\pdiff{\tau_{r}}{x^{\mu}}\pdiff{x_{p}^{\nu}}{\tau_{r}}g_{\nu\eta}v_{p}^{\eta}
	                          +\gamma^{\mu}\pdiff{\tau_{r}}{x^{\mu}}X^{\nu}g_{\mu\nu}\pdiff{v_{p}^{\eta}}{\tau_{r}} \nn \\
	                       &= v_{p}-\nabla\tau_{r}\paren{v_{p}\cdot v_{p} - X\cdot\dot{v}_{p}} \nn \\
	                       &= v_{p}-\bfrac{X}{v_{p}\cdot X}\paren{v_{p}\cdot v_{p} - X\cdot\dot{v}_{p}} \nn \\
	                       &= v_{p}+\bfrac{X\paren{X\cdot\dot{v}_{p}-1}}{X\cdot v_{p}} 
\end{align}
Now we can calculate $\nabla A$

\begin{align}\label{eq8_54}
	\nabla A &= \bfrac{q}{4\pi}\paren{\bfrac{\nabla v_{p}}{X\cdot v_{p}}-\bfrac{1}{\paren{X\cdot v_{p}}^{2}}\nabla\paren{X\cdot v_{p}}v_{p}} \nn \\
	         &= \bfrac{q}{4\pi}\paren{\bfrac{X\dot{v}_{p}}{\paren{X\cdot v_{p}}^{2}}-\bfrac{1}{\paren{X\cdot v_{p}}^{2}}\paren{v_{p}+\bfrac{X\paren{X\cdot\dot{v}_{p}-1}}{X\cdot v_{p}}}v_{p}} \nn \\
	         &= \bfrac{q}{4\pi\paren{X\cdot v_{p}}^{2}}\paren{X\dot{v}_{p}-1-\bfrac{Xv_{p}\paren{X\cdot\dot{v}_{p}-1}}{X\cdot v_{p}}} \nn \\
	         &= \bfrac{q}{4\pi\paren{X\cdot v_{p}}^{2}}\paren{X\W\dot{v}_{p}+\bfrac{X\W v_{p}-\paren{X\cdot\dot{v}_{p}}X\W v_{p}}{X\cdot v_{p}}} \nn \\
	         &= \bfrac{q}{4\pi\paren{X\cdot v_{p}}^{3}}\paren{X\W v_{p}+ \paren{X\W\dot{v}_{p}}\paren{X\cdot v_{p}}-\paren{X\cdot\dot{v}_{p}}\paren{X\W v_{p}}}
\end{align}
Note that equation~\ref{eq8_54} is a pure bivector so that $\nabla\cdot A = 0$ and $\nabla\W A = \nabla A$ so that $F = \nabla A$.  Now expand
$\paren{X\W\dot{v}_{p}}\paren{X\cdot v_{p}}-\paren{X\cdot\dot{v}_{p}}\paren{X\W v_{p}}$ using the definitons of $\W$ and $\cdot$ for vectors to get
\be\label{eq8_55}
	F = \bfrac{q}{4\pi\paren{X\cdot v_{p}}^{3}}\paren{X\W v_{p}+ \half X\paren{\dot{v}_{p}\W v_{p}}X}.
\ee
The first term in equation~\ref{eq8_55} falls of as $\bfrac{1}{r^{2}}$ and is the static field term.  The second term falls off as $\bfrac{1}{r}$ and is the 
radiation field term so that
\be
	F_{\mbox{rad}} = \bfrac{qX\paren{\dot{v}_{p}\W v_{p}}X}{8\pi\paren{X\cdot v_{p}}^{3}}.
\ee
