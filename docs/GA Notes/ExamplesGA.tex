\chapter{Examples of Geometric Algebra}
\section{Quaternions}
Any multivector $A \in \GA{3,0}$ may be written as
\be
A = \alpha + a + B + \beta I
\ee 
where $\alpha,\beta \in \Re$, $a \in \Vsp\lp 3,0 \rp$, $B$ is a bivector, and
$I$ is the unit pseudoscalar. The quaternions are the multivectors of even grades
\be
A = \alpha + B
\ee
$B$ can be represented as
\be
B = \alpha{\bf i}+\beta{\bf j} + \gamma{\bf k}
\ee
where ${\bf i} = e_{2}e_{3}$, ${\bf j} = e_{1}e_{3}$, and ${\bf k} = e_{1}e_{2}$, and
\be
{\bf i}^{2} = {\bf j}^{2} = {\bf k}^{2} = {\bf ijk} = -1.
\ee
The quaternions form a subalgebra of $\GA{3,0}$ since the geometric product of any
two quaternions is also a quaternion since the geometric product of two even grade
multivector components is a even grade multivector. For example the product of two
grade 2 multivectors can only consist of grades 0, 2, and 4, but in $\GA{3,0}$ we can only have grades 0 and 2 since the highest possible grade is 3.
\section{Spinors}
The general definition of a spinor is a multivector, $\psi \in \GA{p,q}$, such that
$\psi v \psi^{\R} \in \Vsp\paren{p,q} \ \forall v \in \Vsp\paren{p,q}$. Practically speaking a spinor is the
composition of a rotation and a dilation (stretching or shrinking) of a vector.  Thus we can write
\be 
	\psi v \psi^{\R} = \rho R v R^{\R}
\ee
where $R$ is a rotor $\paren{RR^{\R}=1}$. Letting $U = R^{\R}\psi$ we must solve
\be\label{eq82}
	UvU^{\R} = \rho v
\ee
$U$ must generate a pure dilation.  The most general form for $U$ based on the fact that the l.h.s of 
equation~\ref{eq82} must be a vector is
\be
	U = \alpha+\beta I
\ee
so that
\be
UvU^{\R} = \alpha^{2}v+\alpha\beta\paren{Iv+vI^{\R}}+\beta^{2}IvI^{\R} = \rho v
\ee
Using $vI^{\R} = \paren{-1}^{\frac{\paren{n-1}\paren{n-2}}{2}}Iv$, $vI^{\R} = \paren{-1}^{n-1}I^{\R}v$,
and $II^{\R} = \paren{-1}^{q}$ we get
\be
	\alpha^{2}v+\alpha\beta\paren{1+\paren{-1}^{\frac{\paren{n-1}\paren{n-2}}{2}}}Iv
	+\paren{-1}^{n+q-1}\beta^{2}v = \rho v
\ee
If $\bfrac{\paren{n-1}\paren{n-2}}{2}$ is even $\beta = 0$ and $\alpha \ne 0$, otherwise $\alpha,\beta \ne 0$.
For the odd case 
\be
	\psi = R\paren{\alpha + \beta I}
\ee
where $\rho = \alpha^{2}+\paren{-1}^{n+q-1}\beta^{2}$. In the case of $\GA{1,3}$ (relativistic space time) we have 
$\rho = \alpha^{2}+\beta^{2}$, $\rho > 0$.
\section{Geometric Algebra of the Minkowski Plane}
Because of Relativity and QM the Geometric Algebra of the Minkowski Plane is very important for physical applications of Geometric Algebra so we will treat it in detail.

Let the orthonormal basis vectors for the plane be $\gO$ and $\gl$ where $\gO^{2}=-\gl^{2}=1$.\footnote{$I = \gO\gl$} Then the geometric product of
two vectors $a=a_{0}\gO+a_{1}\gl$ and $b=b_{0}\gO+b_{1}\gl$ is
\begin{align}
	ab &= \paren{a_{0}\gO+a_{1}\gl}\paren{b_{0}\gO+b_{1}\gl} \\
	   &= a_{0}b_{0}\gO^{2}+a_{1}b_{1}\gl^{2}+\paren{a_{0}b_{1}-a_{1}b_{0}}\gO\gl \\
	   &= a_{0}b_{0}-a_{1}b_{1}+\paren{a_{0}b_{1}-a_{1}b_{0}}I
\end{align}
so that
\be
	a\cdot b = a_{0}b_{0}-a_{1}b_{1}
\ee
and
\be
	a\w b = \paren{a_{0}b_{1}-a_{1}b_{0}}I
\ee
and
\be
	I^{2} = \gO\gl\gO\gl = -\gO^{2}\gl^{2} = 1
\ee
Thus
\begin{align}
	e^{\alpha I} &= \sum_{i=0}^{\infty}\bfrac{\alpha^{i}I^{i}}{i!} \\
				 &= \sum_{i=0}^{\infty}\bfrac{\alpha^{2i}}{\paren{2i}!}+
 						\sum_{i=0}^{\infty}\bfrac{\alpha^{2i+1}I}{\paren{2i+1}!} \\
			     &= \cosh\paren{\alpha}+\sinh\paren{\alpha}I
\end{align}
since $I^{2i} = 1$.

In the Minkowski plane all vectors of the form $a_{\pm} = \alpha\paren{\gO\pm\gl}$ are null $\paren{a_{\pm}^{2}=0}$. One
question to answer are there any vectors $b_{\pm}$ such that $a_{\pm}\cdot b_{\pm} = 0$ that are not parallel to
$a_{\pm}$.
\begin{center}
\begin{tabular}{c}
$a_{\pm}\cdot b_{\pm} = \alpha\paren{b_{0}^{\pm}\mp b_{1}^{\pm}} = 0 $ \\
$b_{0}^{\pm} \mp b_{1}^{\pm} = 0 $ \\
$b_{0}^{\pm} = \pm b_{1}^{\pm} $
\end{tabular}
\end{center}
Thus $b_{\pm}$ must be proportional to $a_{\pm}$ and the are no vectors in the space that can be constructed that are normal to $a_{\pm}$. Thus there are no non-zero bivectors, $a \w b$, such that $\paren{a\w b}^{2}=0$.
Conversely, if $a \w b \ne 0$ then $\paren{a \w b}^{2} > 0$.

Finally for the condition that there always exist two orthogonal vectors $e_{1}$ and $e_{2}$ such that
\be
	a\w b = e_{1}e_{2}
\ee
we can state that neither $e_{1}$ nor $e_{2}$ can be null.
\section{Lorentz Transformation}
We now have all the tools needed to derive the Lorentz transformation with Geometric Algebra. Consider a two
dimensional time-like plane with with coordinates $t$\footnote{We let the speed of light $c=1$.} 
and $x_{1}$ and basis vectors $\gO$ and $\gl$. Then a general space-time vector in the plane is given by
\be
	x = t\gO+x_{1}\gl = t'\gO'+x'_{1}\gl'
\ee 
where the basis vectors of the two coordinate systems are related by
\be
	\gamma'_{\mu} = R\gamma_{\mu}R^{\R}\ \mu = 0,1
\ee
and $R$ is a Minkowski plane rotor
\be
	R = \sinhf{\alphah}+\coshf{\alphah}\gl\gO
\ee
so that
\be
	R\gO R^{\R} = \coshf{\alpha}\gO+\sinhf{\alpha}\gl
\ee
and
\be
	R\gl R^{\R} = \coshf{\alpha}\gl+\sinhf{\alpha}\gO
\ee
Now consider the special case that the primed coordinate system is moving with velocity $\beta$ in the
direction of $\gl$ and the two coordinate systems were coincident at time $t = 0$. Then $x_{1} = \beta t$ 
and $x'_{1} = 0$ so we may write
\be
	t\gO+\beta t\gl = t'R\gO R^{\R}
\ee
\be
	\bfrac{t}{t'}\paren{\gO+\beta\gl} = \coshf{\alpha}\gO+\sinhf{\alpha}\gl 
\ee
Equating components gives
\be\label{eq106}
	\coshf{\alpha} = \bfrac{t}{t'}\
\ee
\be\label{eq107}
	\sinhf{\alpha} = \bfrac{t}{t'}\beta
\ee
Solving for $\alpha$ and $\bfrac{t}{t'}$ in equations~\ref{eq106} and \ref{eq107} gives
\be
	\tanhf{\alpha} = \beta
\ee
\be
	\bfrac{t}{t'} = \gamma = \bfrac{1}{\sqrt{1-\beta^{2}}}
\ee
Now consider the general case of $x,t$ and $x',t'$ giving
\begin{align}
t\gO+x\gl &= t'R\gO R^{\R}+x'R\gl R^{\R} \\
          &= t'\gamma\paren{\gO+\beta\gl}+x'\gamma\paren{\gl+\beta\gO}
\end{align}
Equating basis vector coefficients recovers the Lorentz transformation
\be
\begin{array}{c}
	t = \gamma\paren{t'+\beta x'} \\
	x = \gamma\paren{x'+\beta t'} 
\end{array}
\ee
