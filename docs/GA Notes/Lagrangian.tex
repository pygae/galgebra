\chapter[Lagrangian and Hamiltonian Methods]{Lagrangian and Hamiltonian Methods\footnote{This chapter follows ``A Multivector 
Derivative Approach to Lagrangian Field Theory,'' by A. Lasenby, C. Doran, and S. Gull, Feb. 9, 1993 available at http://www.mrao.cam.ac.uk/~cjld1/pages/publications.htm}}

\section{Lagrangian Theory for Discrete Systems}
\subsection{The Euler-Lagrange Equations}
Let a system be described by multivector variables $X_{i}$, $i=1,\dots,m$.  The Lagrangian $L$ is a scalar valued function of 
the $X_{i}$, $\dot{X}_{i}$ (here the dot refers to the time derivative), and possibly the time, $t$.  The action for the system, $S$,
over a time interval is given by the integral
\be
	S \equiv \int_{t_{1}}^{t_{2}}dt\f{L}{X_{i},\dot{X}_{i},t}.
\ee
The statement of the principal of least action is that the variation of the action $\delta S=0$.  The rigorous definition of $\delta S=0$ is
let 
\be
	\f{X'_{i}}{t} = \f{X_{i}}{t}+\epsilon\f{Y_{i}}{t}
\ee
where $\f{Y_{i}}{t}$ is an arbitrary differentiable multivector function of time except that $\f{Y_{i}}{t_{1}} = \f{Y_{i}}{t_{2}}=0$. Then
\be
	\delta S \equiv \eval{\deriv{S}{\epsilon}}{\epsilon=0} = 0.
\ee
Then
\begin{align}
    \f{L}{X'_{i},\dot{X}'_{i},t} &= \f{L}{X_{i}+\epsilon Y_{i},\dot{X}_{i}+\epsilon\dot{Y}_{i},t} \label{eq7_4a} \\
                                 &= \f{L}{X_{i},\dot{X}_{i},t}+\epsilon\sum_{i=1}^{m}\paren{Y_{i}*\pD{X_{i}}{L}+\dot{Y}_{i}*\pD{\dot{X}_{i}}{L}} \label{eq7_4b}\\
                               S &= \int_{t_{1}}^{t_{2}} dt\paren{\f{L}{X_{i},\dot{X}_{i},t}+\epsilon\sum_{i=1}^{m}\paren{Y_{i}*\pD{X_{i}}{L}
                                    +\dot{Y}_{i}*\pD{\dot{X}_{i}}{L}} } \\
	\eval{\deriv{S}{\epsilon}}{\epsilon=0} &= \int_{t_{1}}^{t_{2}}dt\sum_{i=1}^{m}\paren{Y_{i}*\partial_{X_{i}}L+\dot{Y}_{i}*\partial_{\dot{X}_{i}}L} \label{eq7_4} \\
	                                       &= \int_{t_{1}}^{t_{2}}dt\sum_{i=1}^{m}Y_{i}*\paren{\partial_{X_{i}}L-\deriv{}{t}\paren{\partial_{\dot{X}_{i}}L}} \label{eq7_5}                               
\end{align}
where we use the definition of the multivector derivative to go from equation~\ref{eq7_4a} to equation~\ref{eq7_4b} and then use integration by parts with respect
to time to go from equation~\ref{eq7_4} to equation~\ref{eq7_5}.  Since in equation~\ref{eq7_5} the $Y_{i}$'s are arbitrary $\delta S=0$ implies that the 
Lagrangian equations of motion are
\be
	\partial_{X_{i}}L-\deriv{}{t}\paren{\partial_{\dot{X}_{i}}L} = 0,\hspace{12pt}\forall i=1,\dots,m. \label{eq7_6}
\ee
The multivector derivative insures that there are as many equations as there are grades present in the $X_{i}$, which implies there are the same number of equations
as there are degrees of freedom in the system.

\subsection{Symmetries and Conservation Laws}\label{SandC}
Consider a scalar parametrised transformation of the dynamical variables
\be
	X'_{i} = \f{X'_{i}}{X_{i},\alpha},
\ee
where $\f{X'_{i}}{X_{i},0} = X_{i}$. Now define
\be
	\var{X_{i}} \equiv \eval{\deriv{X'_{i}}{\alpha}}{\alpha=0}.
\ee
and a transformed Lagrangian 
\be
	\f{L'}{X_{i},\dot{X}_{i},t} \equiv \f{L}{X'_{i},\dot{X}'_{i},t}.
\ee
Then
\begin{align}
	\eval{\deriv{L'}{\alpha}}{\alpha=0} &= \sum_{i=1}^{m}\paren{\var{X_{i}}*\partial_{X'_{i}}L'+\var{\dot{X}_{i}}*\partial_{\dot{X}'_{i}}L'} \nonumber \\
	                                    &= \sum_{i=1}^{m}\paren{\var{X_{i}}*\partial_{X'_{i}}L'
	                                       +\deriv{}{t}\paren{\var{X_{i}}*\partial_{\dot{X}'_{i}}L'}
	                                       -\var{X_{i}}*\deriv{}{t}\paren{\partial_{\dot{X}'_{i}}L'}} \nonumber \\
	                                    &= \sum_{i=1}^{m}\paren{\var{X_{i}}*\paren{\partial_{X'_{i}}L'
	                                       -\deriv{}{t}\paren{\partial_{\dot{X}'_{i}}L'}}
	                                       +\deriv{}{t}\paren{\var{X_{i}*\partial_{\dot{X}'_{i}}L'}}}. \label{eq7_13}
\end{align}
If the $X'_{i}$'s satisfy equation~\ref{eq7_6} equation~\ref{eq7_13} can be rewritten as
\be
	\eval{\deriv{L'}{\alpha}}{\alpha=0} = \deriv{}{t}\sum_{i=1}^{m}\paren{\var{X_{i}}*\partial_{\dot{X}_{i}}L}.\label{eq7_14}
\ee
Noether's theorem is -
\be\label{eqPPNT}
    \eval{\deriv{L'}{\alpha}}{\alpha=0}\hspace{-12pt} = 0 \implies \sum_{i=1}^{m}\paren{\var{X_{i}}*\partial_{\dot{X}_{i}}L} = \mbox{conserved quantity}
\ee

From D\& L -

``If the transformation is a symmetry of the Lagrangian, then $L'$ is independent of $\alpha$.  In this case we immediately establish that a conjugate quantity
is conserved.  That is, symmetries of the Lagrangian produce conjugate conserved quantities.  This is Noether's theorem, and it is valuable for extracting
conserved quantities from dynamical systems.  The fact that the derivation of equation~\ref{eq7_14} assumed the equations of motion were satisfied means that
the quantity is conserved `on-shell'.  Some symmetries can also be extended `off-shell', which becomes an important issue in quantum and super symmetric systems.''

A more general treatment of symmetries and conservation is possible if we do not limit ourselves to a scalar parametrization.  Instead let
$X'_{i} = \f{X'_{i}}{X_{i},M}$ where $M$ is a multivector parameter. Then let
\be
	L' = \f{L}{X'_{i},\dot{X}'_{i},t}
\ee
and calculate the multivector derivative of $L'$ with respect to $M$ using the chain rule (summation convention for repeated indices) first noting that
\be
	\pdiff{}{t}\paren{\paren{\paren{A*\partial_{M}}X'_{i}}*\partial_{\dot{X}'_{i}}L'} = \paren{\paren{A*\partial_{M}}\dot{X}'_{i}}*\partial_{\dot{X}'_{i}}L'
		+\paren{\paren{A*\partial_{M}}X'_{i}}*\pdiff{}{t}\paren{\partial_{\dot{X}'_{i}}L'}
\ee
and then calculating
\begin{align}
	\paren{A*\partial_{M}}L' &= \paren{\paren{A*\partial_{M}}X'_{i}}*\partial_{X'_{i}}L'
	                            +\paren{\paren{A*\partial_{M}}\dot{X}'_{i}}*\partial_{\dot{X}'_{i}}L' \nonumber \\
	                       &= \paren{\paren{A*\partial_{M}}X'_{i}}*\paren{\partial_{X'_{i}}L'-\pdiff{}{t}\paren{\partial_{\dot{X}'_{i}}L'}}
	                          +\pdiff{}{t}\paren{\paren{\paren{A*\partial_{M}}X'_{i}}*\partial_{\dot{X}'_{i}}L'}.
\end{align}
If we assume that the $X'_{i}$'s satisfy the equations of motion we have
\be
	\paren{A*\partial_{M}}L' = \pdiff{}{t}\paren{\paren{\paren{A*\partial_{M}}X'_{i}}*\partial_{\dot{X}'_{i}}L'}\label{eq7_22}
\ee
and differentiating equation~\ref{eq7_22} with respect to $A$ (use equation~\ref{eq6_40a}) gives
\begin{align}
	\partial_{M}L' &= \pdiff{}{t}\paren{\partial_{A}\paren{\paren{A*\partial_{M}}X'_{i}}*\partial_{\dot{X}'_{i}}L'} \nonumber \\
	               &= \pdiff{}{t}\paren{\paren{\partial_{M}X'_{i}}*\partial_{\dot{X}'_{i}}L'}. \label{eq7_25a} 
\end{align}
Equation~\ref{eq7_25a} is a generalization of Noether's theorem since if $\partial_{M}L'= 0$ then the parametrization $M$ is a symmetry of the Lagrangian
and $\paren{\partial_{M}X'_{i}}*\partial_{\dot{X}'_{i}}L'$ is a conserved quantity.

\subsection{Examples of Lagrangian Symmetries}

\subsubsection{Time Translation}

Consider the symmetry of time translation
\be
	\f{X'_{i}}{t,\alpha} = \f{X_{i}}{t+\alpha}
\ee 
so that 
\be
	\var{X_{i}} = \eval{\deriv{X'_{i}}{\alpha}}{\alpha=0} = \dot{X}_{i},
\ee
and
\begin{align}
	\left .\deriv{L}{\alpha}\right |_{\alpha=0} &= \deriv{}{t}\sum_{i=1}^{m}\paren{\dot{X}_{i}*\partial_{\dot{X}_{i}}L} \\
	0 &= \deriv{}{t}\paren{\sum_{i=1}^{m}\paren{\dot{X}_{i}*\partial_{\dot{X}_{i}}L}-L}.
\end{align}
The conserved quantity is the Hamiltonian
\be
	H = \sum_{i=1}^{m}\paren{\dot{X}_{i}*\partial_{\dot{X}_{i}}L}-L.
\ee
In terms of the generalized momenta
\be
	P_{i} = \partial_{\dot{X}_{i}}L,
\ee
so that
\be
	H = \sum_{i=1}^{m}\paren{\dot{X}_{i}*P_{i}}-L.
\ee


\subsubsection{Central Forces}
Let the Lagrangian variables be $x_{i}$ the vector position of the $i^{th}$ particle in an ensemble of $N$ particles with a Lagrangian of the form
\be
    L = \sum_{i=1}^{N}\half m_{i}\dot{x}_{i}^{2} - \sum_{i=1}^{N}\sum_{j<i}^{N}\f{V_{ij}}{\abs{x_{i}-x_{j}}}
\ee
which represent a classical system with central forces between each pair of particles.

First consider a translational invariance so that
\be
    x'_{i} = x_{i}+\alpha c
\ee
where $\alpha$ is a scalar parameter and $c$ is a constant vector.  Then
\be
    \delta x'_{i} = c
\ee
and
\be
    L' = L
\ee
so that the conserved quantity is (equation~\ref{eqPPNT})
\begin{align}
    \sum_{i=1}^{N}\delta x_{i} * \partial_{\dot{x_{i}}} L &= c * \sum_{i=1}^{N} \partial_{\dot{x_{i}}} L \nonumber \\
    c\cdot p &= c \cdot \sum_{i=1}^{N} m_{i}\dot{x}_{i} \nonumber \\
    p &= \sum_{i=1}^{N} m_{i}\dot{x}_{i}
\end{align}
since $c$ is an arbitrary vector the vector $p$ is also conserved and is the linear momentum of the system.

Now consider a rotational invariance where
\be
    x'_{i} = e^{\alpha B/2}x_{i}e^{-\alpha B/2}
\ee
where $B$ is an arbitrary normalized ($B^{2}=-1$) bivector in 3-dimensions and $\alpha$ is the scalar 
angle of rotation.  Then again $L' = L$ since rotations leave $\dot{x}_{i}^{2}$ and $\abs{x_{i}-x_{j}}$
unchanged and
\begin{align}
    \deriv{x'_{i}}{\alpha} &= \half\paren{Be^{\alpha B/2}x_{i}e^{-\alpha B/2}-e^{\alpha B/2}x_{i}e^{-\alpha B/2}B} \nonumber \\
    \delta x'_{i} &= \half\paren{Bx_{i}-x_{i}B} \nonumber \\
                  &= B\cdot x_{i}.
\end{align}
Remember that since $B\cdot x_{i}$ is a vector and the scalar product ($*$) of two vectors is the dot product we have for a 
conserved quantity
\begin{align}
  \sum_{i=1}^{N}\paren{B\cdot x_{i}}\cdot\paren{\partial_{\dot{x}_{i}}L} 
           &= \sum_{i=1}^{N}m_{i}\paren{B\cdot x_{i}}\cdot\dot{x}_{i} \label{eq7_35_s1}\\
           &= B\cdot \sum_{i=1}^{N}m_{i}\paren{x_{i}\W\dot{x}_{i}} \label{eq7_35_s2} \\
           &= B\cdot J \\
        J  &= \sum_{i=1}^{N}m_{i}\paren{x_{i}\W\dot{x}_{i}}
\end{align}
where we go from equation~\ref{eq7_35_s1} to equation~\ref{eq7_35_s2} by using the identity in equation~\ref{eq465a}. Then since 
equation~\ref{eq7_35_s1} is conserved for any bivector $B$, the angular momentum bivector, $J$, of the system is conserved.
\section{Lagrangian Theory for Continuous Systems}
For ease of notation we define
\be
	A\lgrad \equiv \dot{A}\dot{\nabla}.\label{eq7_24}
\ee
This is done for the situation that we are left differentiating a group of symbols.  For example consider
\be
	\paren{ABC}\lgrad = \dot{\paren{ABC}}\dot{\nabla}.\label{eq7_25}
\ee
The r.h.s. of equation~\ref{eq7_25} could be ambiguous in that could the overdot only apply to the $B$ variable.  Thus we will use the convention
of equation~\ref{eq7_24} to denote differentiation of the group immediately to the left of the derivative.  Another convention we could use to 
denote the same operation is
\be
	\widehat{ABC}\widehat{\nabla} = \paren{ABC}\lgrad
\ee
since using the ``hat'' symbol is unambiguous with respect to what symbols we are applying the differentiation operator to since the ``hat'' can extend
over all the relevant symbols.  
\subsection{The Euler Lagrange Equations}
Let $\f{\psi_{i}}{x}$ be a set of multivector fields and assume the Lagrangian density, $\Lf$, is a scalar function $\f{\Lf}{\psi_{i},\nabla\psi_{i},x}$ so
that the action, $S$, of the continuous system is given by
\be
	S = \int_{V}\abs{dx^{n}}\f{\Lf}{\psi_{i},\nabla\psi_{i},x},\label{eq7_27}
\ee
where $V$ is a compact $n$-dimensional volume.  The equations of motion are given by minimizing $S$ using the standard method of the calculus of 
variations where we define
\be
	\f{\psi'_{i}}{x} = \f{\psi_{i}}{x}+\epsilon\f{\phi_{i}}{x},
\ee
and assume that $\f{\psi_{i}}{x}$ yields an extrema of $S$ and that $\f{\phi_{i}}{x}=0$ for all $x\in\partial V$.  Then to get an extrema we need to define
\be
	\f{S}{\epsilon} = \int_{V}\abs{dx^{n}}\f{\Lf}{\psi_{i}+\epsilon\phi_{i},\nabla\psi_{i}+\epsilon\nabla\phi_{i},x},
\ee
so that $\f{S}{0}$ is an extrema if $\pdiff{S}{\epsilon} = 0$. Let us evaluate $\pdiff{S}{\epsilon}$ (summation convention)
\be
	\eval{\pdiff{S}{\epsilon}}{\epsilon=0} = \int_{V}\abs{dx^{n}}\paren{\paren{\phi_{i}*\partial_{\psi_{i}}}\Lf+\paren{\nabla\phi_{i}*\partial_{\nabla\psi_{i}}}\Lf}.\label{eq7_30}
\ee
Start by reducing the second term in the parenthesis on the r.h.s. of equation~\ref{eq7_30} using RR5 (Appendix~\ref{RRrules})
\begin{align}
	\paren{\nabla\phi_{i}*\partial_{\nabla\psi_{i}}}\Lf &= \grd{\paren{\nabla\phi_{i}}\partial_{\nabla\psi_{i}}}{}\Lf \nonumber \\
	                                                    &= \grd{\paren{\nabla\phi_{i}}\partial_{\nabla\psi_{i}}\Lf}{} \nonumber \\
	                                                    &= \grd{\nabla\paren{\phi_{i}\partial_{\nabla\psi_{i}}\Lf}
	                                                       -\widehat{\nabla}\phi_{i}\paren{\widehat{\partial_{\nabla\psi_{i}}\Lf}}}{} \nonumber \\
	                                                    &= \nabla\cdot\grd{\phi_{i}\partial_{\nabla\psi_{i}}\Lf}{1}
	                                                       -\grd{\widehat{\nabla}\phi_{i}\paren{\widehat{\partial_{\nabla\psi_{i}}\Lf}}}{} \nonumber \\
	                                                    &= \nabla\cdot\grd{\phi_{i}\partial_{\nabla\psi_{i}}\Lf}{1}
	                                                       -\grd{\phi_{i}\paren{\partial_{\nabla\psi_{i}}\Lf}\lgrad}{}  \nonumber \\   
	                                                    &= \nabla\cdot\grd{\phi_{i}\partial_{\nabla\psi_{i}}\Lf}{1}
	                                                       -\phi_{i}*\paren{\paren{\partial_{\nabla\psi_{i}}\Lf}\lgrad}.
\end{align}
So that equation~\ref{eq7_30} becomes
\begin{align}
	\eval{\pdiff{S}{\epsilon}}{\epsilon=0} &= 
	                       \int_{V}\abs{dx^{n}}\phi_{i}*\paren{\partial_{\psi_{i}}\Lf-\paren{\partial_{\grad\psi_{i}}\Lf}\lgrad}
	                       +\int_{V}\abs{dx^{n}}\nabla\cdot\grd{\phi_{i}\partial_{\nabla\psi_{i}}\Lf}{1} \nonumber \\
	                    &= \int_{V}\abs{dx^{n}}\phi_{i}*\paren{\partial_{\psi_{i}}\Lf-\paren{\partial_{\grad\psi_{i}}\Lf}\lgrad}
	                       +\int_{\partial V}\abs{dS^{n-1}}n\cdot\grd{\phi_{i}\partial_{\nabla\psi_{i}}\Lf}{1} \nonumber \\
	                    &= \int_{V}\abs{dx^{n}}\phi_{i}*\paren{\partial_{\psi_{i}}\Lf-\paren{\partial_{\grad\psi_{i}}\Lf}\lgrad} \label{eq7_32}.
\end{align}
The $\int_{V}\abs{dx^{n}}\nabla\cdot\grd{\phi_{i}\partial_{\nabla\psi_{i}}\Lf}{1}$ term is found to be zero by using the generalized divergence theorem (equation~\ref{eq_divth}) and the
fact that the $\phi_{i}$'s are zero on $\partial V$.  If $\phi_{i}$ is a pure $r$-grade multivector we have by the properties of
the scalar product the following Lagrangian field equations
\be
	\grade{\partial_{\psi_{i}}\Lf-\paren{\partial_{\grad\psi_{i}}\Lf}\lgrad}{r} = 0\label{eq7_33a}
\ee
or
\be
	\grade{\partial_{\psi_{i}^{\R}}\Lf-\nabla\paren{\partial_{\paren{\grad\psi_{i}}^{\R}}\Lf}}{r} = 0.\label{eq7_34b}
\ee
For the more general case of a $\phi_{i}$ being a mixed grade multivector the Lagrangian field equations are
\be
	\partial_{\psi_{i}}\Lf-\paren{\partial_{\grad\psi_{i}}\Lf}\lgrad = 0\label{eq7_33}
\ee
or
\be
	\partial_{\psi_{i}^{\R}}\Lf-\nabla\paren{\partial_{\paren{\grad\psi_{i}}^{\R}}\Lf} = 0.\label{eq7_34}
\ee
Note that since equation~\ref{eq7_33} is true for $\psi_{i}$ being any kind of multivector field we have derived the field equations for vectors, tensors (antisymmetric), spinors, or any combination thereof.
\subsection{Symmetries and Conservation Laws}
We proceed as in section~\ref{SandC} and let $\psi'_{i} = \f{\psi'_{i}}{\psi_{i},M}$ where $M$ is a multivector parameter. Then
\be
	\f{\Lf'}{\psi_{i},\nabla\psi_{i}} \equiv \f{\Lf}{\psi'_{i},\nabla\psi'_{i}}
\ee
and using equation~\ref{eq_chainrule} and RR5
\begin{align}
	\paren{A*\partial_{M}}\Lf' &= \paren{\paren{A*\partial_{M}}\psi'_{i}}*\paren{\partial_{\psi'_{i}}\Lf'}
	                      +\paren{\paren{A*\partial_{M}}\nabla\psi'_{i}}*\paren{\partial_{\nabla\psi'_{i}}\Lf'} \nonumber \\
	                   &= \paren{\paren{A*\partial_{M}}\psi'_{i}}*\paren{\partial_{\psi'_{i}}\Lf'}
	                      +\grd{\paren{\paren{A*\partial_{M}}\nabla\psi'_{i}}\paren{\partial_{\nabla\psi'_{i}}\Lf'}}{} \nonumber \\
	                   &= \paren{\paren{A*\partial_{M}}\psi'_{i}}*\paren{\partial_{\psi'_{i}}\Lf'}
	                      +\grd{\nabla\paren{\paren{\paren{A*\partial_{M}}\psi'_{i}}\paren{\partial_{\nabla\psi'_{i}}\Lf'}}
	                      -\widehat{\nabla}\paren{\paren{\paren{A*\partial_{M}}\psi'_{i}}\paren{\widehat{\partial_{\nabla\psi'_{i}}\Lf'}}}}{} \nonumber \\
	                   &= \paren{\paren{A*\partial_{M}}\psi'_{i}}*\paren{\partial_{\psi'_{i}}\Lf'}
	                      +\nabla\cdot\grd{\paren{\paren{A*\partial_{M}}\psi'_{i}}\paren{\partial_{\nabla\psi'_{i}}\Lf'}}{1}
	                      -\grd{\paren{\paren{A*\partial_{M}}\psi'_{i}}\paren{\partial_{\nabla\psi'_{i}}\Lf'}\lgrad}{} \nonumber \\
	                   &= \paren{\paren{A*\partial_{M}}\psi'_{i}}*\paren{\partial_{\psi'_{i}}\Lf'}
	                      +\nabla\cdot\grd{\paren{\paren{A*\partial_{M}}\psi'_{i}}\paren{\partial_{\nabla\psi'_{i}}\Lf'}}{1}
	                      -\paren{\paren{A*\partial_{M}}\psi'_{i}}*\paren{\partial_{\nabla\psi'_{i}}\Lf'}\lgrad \nonumber \\
	                   &= \paren{\paren{A*\partial_{M}}\psi'_{i}}*\paren{\paren{\partial_{\psi'_{i}}\Lf'}-\paren{\partial_{\nabla\psi'_{i}}\Lf'}\lgrad}
	                      +\nabla\cdot\grd{\paren{\paren{A*\partial_{M}}\psi'_{i}}\paren{\partial_{\nabla\psi'_{i}}\Lf'}}{1}
\end{align}
and if the Euler-Lagrange equations are satisfied we have
\be
    \paren{A*\partial_{M}}\Lf' = \nabla\cdot\grd{\paren{\paren{A*\partial_{M}}\psi'_{i}}\paren{\partial_{\nabla\psi'_{i}}\Lf'}}{1}
\ee
and by equation~\ref{eq6_40a}
\be
	\partial_{M}\Lf' = \partial_{A}\paren{\nabla\cdot\grd{\paren{\paren{A*\partial_{M}}\psi'_{i}}\paren{\partial_{\nabla\psi'_{i}}\Lf'}}{1}}. \label{eq7_37}
\ee
If $\partial_{M}\Lf'=0$, equation~\ref{eq7_37} is the most general form of Noether's theorem for the scalar valued multivector Lagrangian density.

If in equation~\ref{eq7_37} $M$ is a scalar $\alpha$ and $B$ is a scalar $\beta$ we have
\begin{align}
	\partial_{\alpha}\Lf' &= \partial_{\beta}\paren{\nabla\cdot\grd{\paren{\paren{\beta\partial_{\alpha}}\psi'_{i}}\paren{\partial_{\nabla\psi'_{i}}\Lf'}}{1}} \nonumber \\
	                      &= \nabla\cdot\grd{\paren{\partial_{\alpha}\psi'_{i}}\paren{\partial_{\nabla\psi'_{i}}\Lf'}}{1}.\label{eq7_38}
\end{align}
Thus if $\alpha=0$ in equation~\ref{eq7_38} we have
\be
	\eval{\partial_{\alpha}\Lf'}{\alpha=0} = \nabla\cdot\eval{\grd{\paren{\partial_{\alpha}\psi'_{i}}\paren{\partial_{\nabla\psi'_{i}}\Lf'}}{1}}{\alpha=0}
\ee
which corresponds to an differential transformation ($\partial_{\alpha}\Lf'=0$ is a global transformation). If $\eval{\partial_{\alpha}\Lf'}{\alpha=0}=0$ 
the conserved current is
\be
	j = \eval{\grd{\paren{\partial_{\alpha}\psi'_{i}}\paren{\partial_{\nabla\psi'_{i}}\Lf'}}{1}}{\alpha=0}
\ee
with conservation law
\be
	\nabla\cdot j = 0.
\ee
If $\eval{\partial_{\alpha}\Lf'}{\alpha=0}\ne 0$ we have by the chain rule that $\paren{\f{g}{x}=\eval{\partial_{\alpha}x'}{\alpha=0}}$
\be
	\eval{\partial_{\alpha}\Lf'}{\alpha=0} = \eval{\partial_{\alpha}x'}{\alpha=0}\cdot\eval{\nabla\Lf'}{\alpha=0} = \f{g}{x}\cdot\nabla\Lf
\ee
and consider
\be
	\nabla\cdot\paren{g\Lf} = \paren{\nabla\cdot g}\Lf+g\cdot\nabla\Lf.
\ee
If $\nabla\cdot g=0$ we can write $\eval{\partial_{\alpha}\Lf'}{\alpha=0}$ as a divergence so that
\be
	\nabla\cdot\paren{\grd{\paren{\eval{\partial_{\alpha}\psi'_{i}}{\alpha=0}}\paren{\partial_{\nabla\psi_{i}}\Lf}}{1}-\paren{\eval{\partial_{\alpha}x'}{\alpha=0}}\Lf} = 0
\ee
and
\be
	j = \grd{\paren{\eval{\partial_{\alpha}\psi'_{i}}{\alpha=0}}\paren{\partial_{\nabla\psi_{i}}\Lf}}{1}-\paren{\eval{\partial_{\alpha}x'}{\alpha=0}}\Lf
\ee
is a conserved current if $\nabla\cdot\eval{\partial_{\alpha}x'}{\alpha=0}=0$.

Note that since $\eval{\partial_{\alpha}x'}{\alpha=0}$ is the derivative of a vector with respect to a scalar it itself is a vector.  Thus the 
conserved $j$ is always a vector that could be a linear function of a vector, bivector, etc. depending upon the type of transformation (vector for affine
and bivector for rotation).  However, the conserved quantity of interest may be other than a
vector such as the stress-energy tensor or the angular momentum bivector.  In these cases the conserved vector current must be transformed to the
conserved quantity of interest via the general adjoint transformation $A*\f{\overline{j}}{B} = B*\f{\underline{j}}{A}$. 

\subsection{Space-Time Transformations and their Conjugate Tensors}
The canonical stress-energy tensor is the current associated with the symmetries of space-time translations. As a function of the parameter $\alpha$
we have
\begin{align}
	x' &= x+\alpha n \\
	\f{\psi'_{i}}{x} &= \f{\psi_{i}}{x+\alpha n}
\end{align}
Then
\begin{align}
	\eval{\partial_{\alpha}\Lf'}{\alpha=0} &= \eval{\partial_{\alpha}\f{\Lf}{x+\alpha n}}{\alpha=0} = n\cdot\nabla\Lf = \nabla\cdot\paren{n\Lf}\\
	\eval{\partial_{\alpha}\psi'_{i}}{\alpha=0} &= n\cdot\nabla\psi_{i}
\end{align}
and equation~\ref{eq7_38} becomes
\be
 	\nabla\cdot\paren{n\Lf} = \nabla\cdot\grd{\paren{n\cdot\nabla\psi_{i}}\paren{\partial_{\nabla\psi_{i}}\Lf}}{1}
\ee
so that
\be
	\nabla\cdot\f{\overline{T}}{n} \equiv \nabla\cdot\grd{\paren{n\cdot\nabla\psi_{i}}\paren{\partial_{\nabla\psi_{i}}\Lf}-n\Lf}{1} = 0. \label{eq7_47}
\ee
Thus the conserved current, $\f{\overline{T}}{n}$, is a linear vector function of a vector $n$, a tensor of rank 2.  In order put the stress-energy tensor
into the standard form we need the adjoint, $\f{\underline{T}}{n}$, of $\f{\overline{T}}{n}$ (we are using that the adjoint of the adjoint is the original
linear transformation). 
\begin{align}
	\f{\overline{T}}{n} &= \grd{\paren{n\cdot\nabla\psi_{i}}\paren{\partial_{\nabla\psi_{i}}\Lf}}{1}-n\Lf \\
	                    &= \paren{n\cdot\dot{\nabla}}\grd{\dot{\psi_{i}}\paren{\partial_{\nabla\psi_{i}}\Lf}}{1}-n\Lf.
\end{align}
Using equation~\ref{eq6_34a} we get
\begin{align}
	\f{\underline{T}}{n} &= \partial_{m}\grd{\paren{m\cdot\dot{\nabla}}\grd{\dot{\psi_{i}}\paren{\partial_{\nabla\psi_{i}}\Lf}}{1}n}{}-n\Lf \nonumber \\
	         &= \partial_{m}\paren{m\cdot\dot{\nabla}}\grd{\grd{\dot{\psi_{i}}\paren{\partial_{\nabla\psi_{i}}\Lf}}{1}n}{}-n\Lf \nonumber \\
	         &= \dot{\nabla}\grd{\grd{\dot{\psi_{i}}\paren{\partial_{\nabla\psi_{i}}\Lf}}{1}n}{}-n\Lf \nonumber \\
	         &= \dot{\nabla}\grd{\dot{\psi_{i}}\paren{\partial_{\nabla\psi_{i}}\Lf}n}{}-n\Lf. \label{eq7_70}
\end{align}
From equation~\ref{eq7_47} it follows that
\begin{align}
	0 = \nabla\cdot\f{\overline{T}}{n} &= n\cdot \f{\dot{T}}{\dot{\nabla}}, \\
	\f{\dot{T}}{\dot{\nabla}} &= 0,
\end{align}
or in rectangular coordinates
\be
	\f{\dot{T}}{\dot{\nabla}} = \f{\dot{T}}{e^{\mu}\dot{\pdiff{}{x^{\mu}}}} = \pdiff{}{x^{\mu}}\f{T}{e^{\mu}} 
	                           = \pdiff{T^{\mu\nu}}{x^{\mu}}e_{\nu}.
\ee
Thus in standard tensor notation
\be
	\pdiff{T^{\mu\nu}}{x^{\mu}} = 0,
\ee
so that $\f{T}{n}$ is a conserved tensor.

Now consider rotational transformations and for now assume the $\psi_{i}$ transform as vectors so that
\begin{align}
	x' &= e^{-\frac{\alpha B}{2}}xe^{\frac{\alpha B}{2}} \\
	\f{\psi'_{i}}{x} &= e^{\frac{\alpha B}{2}}\f{\psi_{i}}{x'}e^{-\frac{\alpha B}{2}}, 
\end{align}
Note that we are considering \emph{active} transformations.  The transformation of $x$ to $x'$ maps one distinct
point in space-time to another distinct point in space-time.  We are not considering \emph{passive} transformations
which are only a transformation of coordinates and the position vector does not move in space-time. Because the 
transformation is \emph{active} the sense of rotation of the vector field $\f{\psi}{x}$ is opposite that of the 
rotation of the space-time position vector\footnote{Consider an observer at location $x$ that is rotated to location $x'$. If he is 
rotated in a clockwise sense he observes the vector field to be rotating in a counter clockwise sense.}. Thus
\begin{align}
	\partial_{\alpha}x' &= \half e^{-\frac{\alpha B}{2}}\paren{xB-Bx}e^{\frac{\alpha B}{2}} \nonumber \\
	\eval{\partial_{\alpha}x'}{\alpha=0} &= x\cdot B \\	
	\partial_{\alpha}\psi'_{i} &= \half e^{\frac{\alpha B}{2}}\paren{B\f{\psi_{i}}{x'}-\f{\psi_{i}}{x'}B}e^{-\frac{\alpha B}{2}}
	                              +e^{\frac{\alpha B}{2}}\paren{\partial_{\alpha}\f{\psi_{i}}{x'}}e^{-\frac{\alpha B}{2}}  \nonumber \\
	                           &= e^{\frac{\alpha B}{2}}\paren{B\times\f{\psi_{i}}{x'}}e^{-\frac{\alpha B}{2}}
	                              +e^{\frac{\alpha B}{2}}\paren{\partial_{\alpha}x'\cdot\nabla\f{\psi_{i}}{x'}}e^{-\frac{\alpha B}{2}} \\
	\eval{\partial_{\alpha}\psi'_{i}}{\alpha=0} &= B\times \f{\psi_{i}}{x} +\eval{\partial_{\alpha}x'}{\alpha=0}\cdot\nabla\f{\psi_{i}}{x}  \nonumber \\
	                                            &= B\times \psi_{i} +\paren{x\cdot B}\cdot\nabla\psi_{i} =  \psi_{i}\cdot B +\paren{x\cdot B}\cdot\nabla\psi_{i}.          
\end{align}
Thus
\be
	\nabla\cdot\paren{\eval{\partial_{\alpha}x'}{\alpha=0}} = \nabla\cdot\paren{x\cdot B} = -\paren{B\cdot \dot{x}}\cdot\dot{\nabla} = -B\cdot\paren{\dot{x}\W\dot{\nabla}}
	                                           = B\cdot\paren{\nabla\W x} = 0
\ee
since $\nabla\W x = 0$\footnote{Consider
\begin{align*}\nabla\W x &= e^{\nu}\pdiff{}{x^{\nu}}\W x_{\eta}e^{\eta} \\
                         &= \pdiff{}{x^{\nu}}x_{\eta}e^{\nu}\W e^{\eta} \\
                         &= \pdiff{}{x^{\nu}}g_{\eta\mu}x^{\mu}e^{\nu}\W e^{\eta} \\
                         &= \pdiff{x^{\mu}}{x^{\nu}}g_{\eta\mu}e^{\nu}\W e^{\eta} \\
                         &= \delta^{\mu}_{\nu}g_{\eta\mu}e^{\nu}\W e^{\eta} \\
                         &= g_{\eta\nu}e^{\nu}\W e^{\eta} = 0
\end{align*}
since $g_{\eta\nu}$ is symmetric and $e^{\nu}\W e^{\eta} $ is antisymmetric.} the derivative of
the transformed Lagrangian at $\alpha = 0$ is a pure divergence,
\be
	\eval{\partial_{\alpha}\Lf'}{\alpha=0} = \nabla\cdot\paren{\paren{x\cdot B}\Lf}.
\ee
\begin{align}
	\eval{\nabla\cdot\grd{\paren{\partial_{\alpha}\psi'_{i}}\paren{\partial_{\nabla\psi'_{i}}\Lf}}{1}}{\alpha=0} &= 
		\nabla\cdot\grd{\paren{B\times\psi_{i}-\paren{B\cdot x}\cdot\nabla\psi_{i}}\paren{\partial_{\nabla\psi_{i}}\Lf}}{1}
\end{align}
\begin{align}
	\nabla\cdot\f{\overline{J}}{B} &\equiv \eval{\nabla\cdot\grd{\paren{\partial_{\alpha}\psi'_{i}}\paren{\partial_{\nabla\psi'_{i}}\Lf}}{1}}{\alpha=0}
	                                  -\eval{\partial_{\alpha}\Lf'}{\alpha=0} \nonumber \\
	                               &= \nabla\cdot\paren{\grd{\paren{B\times\psi_{i}-\paren{B\cdot x}\cdot\nabla\psi_{i}}\paren{\partial_{\nabla\psi_{i}}\Lf}}{1}-
	                                  \paren{x\cdot B}\Lf}
\end{align}
\be\label{eq7_67}
	\f{\overline{J}}{B} = \grd{\paren{B\times\psi_{i}-\paren{B\cdot x}\cdot\nabla\psi_{i}}\paren{\partial_{\nabla\psi_{i}}\Lf}}{1}-\paren{x\cdot B}\Lf.
\ee
By Noether's theorem $\dot{\nabla}\cdot\f{\dot{\overline{J}}}{B} = 0$ where $\f{\overline{J}}{B}$ is a conserved vector. So that
\begin{equation*}
	\dot{\nabla}\cdot\f{\dot{\overline{J}}}{B} = 0 \quad \Rightarrow \quad \f{\dot{\underline{J}}}{\dot{\nabla}}\cdot B = 0 \mbox{ for all } B 
	\quad\Rightarrow\quad \f{\dot{\underline{J}}}{\dot{\nabla}} = 0
\end{equation*}
The adjoint functon $\f{\underline{J}}{n}$ is, therefore a conserved bivector-valued function of position 
(we can use $\cdot$ instead of $*$ since all the grades match up correctly).

Now consider the coordinate (tensor) representation of $\f{\dot{\underline{J}}}{\dot{\nabla}} = 0$
\begin{align}
	n &= n_{\gamma}\bm{e}^{\gamma} \\
	\f{\underline{J}}{n} &= J^{\mu\nu\gamma}n^{\gamma}\bm{e}_{\mu}\W \bm{e}_{\nu} \\
	\f{\dot{\underline{J}}}{\dot{\nabla}} &= \pdiff{J^{\mu\nu\gamma}}{x^{\gamma}}\bm{e}_{\mu}\W \bm{e}_{\nu} = 0 \\
	\pdiff{J^{\mu\nu\gamma}}{x^{\gamma}} &= 0
\end{align}

To compute $\f{\underline{J}}{n}$ note that $A$ is a bivector using equation~\ref{eq6_34a}\footnote{$\f{\overline{F}}{B}=\partial_{A}\grd{\f{\underline{F}}{A}B}{}$} 
to calculate the adjoint of the adjoint
and $\paren{A*\partial_{B}}B = A$\footnote{$\paren{A*\partial_{B}}B = {\ds \lim_{h\rightarrow 0}}\bfrac{B+hA-B}{h}=A$} and 
RR7\footnote{$\paren{\grd{A}{2}\cdot a}\cdot b = \grd{A}{2}\cdot\paren{a\W b}$} we get
\begin{align}
	A*\f{\underline{J}}{n} &= \paren{A*\partial_{B}}\grd{\f{\overline{J}}{B}n}{} \nonumber \\
	                       &= \paren{A*\partial_{B}}\grd{\grd{\paren{B\times\psi_{i}-\paren{B\cdot x}\cdot\nabla\psi_{i}}\paren{\partial_{\nabla\psi_{i}}\Lf}}{1}n
	                          -\paren{x\cdot B}\Lf n}{} \nonumber \\
	                       &= \grd{\paren{A\times\psi_{i}-\paren{A\cdot x}\cdot\nabla\psi_{i}}\paren{\partial_{\nabla\psi_{i}}\Lf n}
	                          -\paren{x\cdot A}\Lf n}{} \nonumber \\
	                       &= \grd{\paren{A\times\psi_{i}-\paren{A\cdot\paren{x\W\nabla}}\psi_{i}}\paren{\partial_{\nabla\psi_{i}}\Lf n}
	                          -\paren{x\cdot A}\Lf n}{}. \label{eq7_68}  
\end{align}
Using RR5\footnote{$\grd{A_{1}\dots A_{k}}{}=\grd{A_{k}A_{1}\dots A_{k-1}}{}$} (Reduction Rule 5 in Appendix~\ref{RRrules}) the first term on the r.h.s.\;of equation~\ref{eq7_68} reduces to
\begin{align}
	\grd{\paren{A\times\psi_{i}}\paren{\partial_{\nabla\psi_{i}}\Lf n}}{} &= \half\grd{\paren{A\psi_{i}-\psi_{i}A}\paren{\partial_{\nabla\psi_{i}}\Lf n}}{} \nonumber \\
		&= \half\grd{A\brkt{\psi_{i}\paren{\partial_{\nabla\psi_{i}}\Lf n}-\paren{\partial_{\nabla\psi_{i}}\Lf n}\psi_{i}}}{}\nonumber \\
		&= \grd{A\paren{\psi_{i}\times\paren{\partial_{\nabla\psi_{i}}\Lf n}}}{} \nonumber \\
		&= A*\grd{\psi_{i}\times\paren{\partial_{\nabla\psi_{i}}\Lf n}}{2}, \label{eq7_69} 
\end{align}
using RR7 the second term reduces to 
\begin{align}
	\grd{\paren{A\cdot\paren{x\W\nabla}}\psi_{i}\paren{\partial_{\nabla\psi_{i}}\Lf n}}{} 
		&= \paren{A\cdot\paren{x\W\dot{\nabla}}}\grd{\dot{\psi}_{i}\partial_{\nabla\psi_{i}}\Lf n}{} \nonumber \\
		&= A*\paren{\paren{x\W\dot{\nabla}}\grd{\dot{\psi}_{i}\partial_{\nabla\psi_{i}}\Lf n}{}}, \label{eq7_70a}
\end{align}
using RR5 again the third term reduces to 
\begin{align}
	\grd{\paren{x\cdot A}\Lf n}{} &= \half\grd{\paren{xA-Ax}\Lf n}{} \nonumber \\
	                              &= \half\grd{A\paren{nx-xn}\Lf}{} \nonumber \\
	                              &= \grd{A\paren{n\W x}\Lf}{} \nonumber \\
	                              &= A*\paren{n\W x}\Lf. \label{eq7_71}
\end{align}
Combining equations~\ref{eq7_69}, \ref{eq7_70a}, and \ref{eq7_71} reduces equation~\ref{eq7_68} to
\be
	\f{\underline{J}}{n} = \grd{\psi_{i}\times\paren{\partial_{\nabla\psi_{i}}\Lf n}}{2}
	                      -\paren{x\W\dot{\nabla}}\grd{\dot{\psi}_{i}\partial_{\nabla\psi_{i}}\Lf n}{}
	                      -\paren{n\W x}\Lf \label{eq7_72}
\ee
where $\f{\underline{J}}{n}$ is the angular momentum bivector for the vector field.  Using equation~\ref{eq7_70a} we can reduce
equation~\ref{eq7_72} to
\be
	\f{\underline{J}}{n} = \f{T}{n}\W x + \grd{\psi_{i}\times\paren{\partial_{\nabla\psi_{i}}\Lf n}}{2}.
\ee

If $\psi_{i}$ is a spinor instead of a vector the transformation law is
\be
	\f{\psi'_{i}}{x} = e^{\frac{\alpha B}{2}}\f{\psi_{i}}{x'}
\ee
so that
\be
	\partial_{\alpha}\psi'_{i} = e^{\frac{\alpha B}{2}}\bfrac{B}{2}\f{\psi_{i}}{x'}+e^{\frac{\alpha B}{2}}\partial_{\alpha}x'\f{\psi_{i}}{x'}
\ee
and
\be
	\eval{\partial_{\alpha}\psi'_{i}}{\alpha=0} = \bfrac{B}{2}\psi_{i}+\paren{x\cdot B}\cdot\nabla\psi_{i}.
\ee
Equations~\ref{eq7_67} and \ref{eq7_72} become
\be
	\f{\overline{J}}{B} = \grd{\paren{\bfrac{B}{2}\psi_{i}-\paren{B\cdot x}\cdot\nabla\psi_{i}}\paren{\partial_{\nabla\psi_{i}}\Lf}}{1}-\paren{x\cdot B}\Lf
\ee
and
\be
	\f{\underline{J}}{n} = \grd{\bfrac{\psi_{i}}{2}\paren{\partial_{\nabla\psi_{i}}\Lf n}}{2}
	                      -\paren{x\W\dot{\nabla}}\grd{\dot{\psi}_{i}\partial_{\nabla\psi_{i}}\Lf n}{}
	                      -\paren{n\W x}\Lf
\ee
where $\f{\underline{J}}{n}$ is the angular momentum bivector for the spinor field.

Since spinors transforms the same as vectors under translations the expression (equation~\ref{eq7_70}) for the stress-energy tensor of a spinor field is the same as for a vector or scalar field, $\f{\underline{T}}{n} = \dot{\nabla}\grd{\dot{\psi_{i}}\paren{\partial_{\nabla\psi_{i}}\Lf}n}{}-n\Lf$, so that
\be
	\mbox{Spinor Field: }\f{\underline{J}}{n} = \f{\underline{T}}{n}\W x + \grd{\bfrac{\psi_{i}}{2}\paren{\partial_{\nabla\psi_{i}}\Lf n}}{2},
\ee
compared to
\be
	\mbox{Vector Field: }\f{\underline{J}}{n} = \f{T}{n}\W x + \grd{\psi_{i}\times\paren{\partial_{\nabla\psi_{i}}\Lf n}}{2}.
\ee

\subsection{Case 1 - The Electromagnetic Field}

Example of Lagrangian densities are the electromanetic field and the spinor field for an electron.  For the electromagnetic field we have
\be
	\mathcal{L} = \half F\cdot F - A\cdot J
\ee
where F = $\nabla\W A$ and using eq~\ref{eq6_15} and eq~\ref{eq6_17}
\begin{align}
		F\cdot F &= \grade{\paren{\nabla \W A}\paren{\nabla \W A}}{} \nonumber \\
		         &= \grade{\half\paren{\nabla A - \paren{\nabla A }^{\R}}\half\paren{\nabla A - \paren{\nabla A }^{\R}}}{} 
		            \nonumber \\
				 &= \bfrac{1}{4}\grade{\paren{\nabla A - \paren{\nabla A}^{\R}}^{2}}{} \nonumber \\
		         &= \bfrac{1}{4}\grade{\nabla A\nabla A-\nabla A\paren{\nabla A}^{\R}
		            -\paren{\nabla A}^{\R}\nabla A + \paren{\nabla A}^{\R}\paren{\nabla A}^{\R}}{} \nonumber \\
		         &= \bfrac{1}{4}\paren{\paren{\nabla A}*\paren{\nabla A} - \paren{\nabla A}*\paren{\nabla A}^{\R}
		           -\paren{\nabla A}^{\R}*\paren{\nabla A}+ \paren{\nabla A}^{\R}*\paren{\nabla A}^{\R}} \nonumber \\
		         &= \half\paren{\paren{\nabla A}*\paren{\nabla A} - \paren{\nabla A}*\paren{\nabla A}^{\R}}
\end{align}
Now calculate using eq~\ref{eq6_47a} and eq~\ref{eq6_51a}
\begin{align}
	\partial_{\paren{\nabla A}^{\R}} F^{2} &= \half\partial_{\paren{\nabla A}^{\R}}
	                                          \paren{\paren{\nabla A}*\paren{\nabla A} - \paren{\nabla A}*\paren{\nabla A}^{\R}}
	                                          \nonumber \\
	                                       &= \paren{\nabla A}^{\R} - \nabla A \nonumber \\
	                                       &= -2\nabla \W A
\end{align}
We also have by eq~\ref{eq6_44a} and that $A^{\R} = A$
\begin{align}
	\partial_{A^{\R}}\paren{A\cdot J} &= \partial_{A^{\R}}\paren{J * A} \nonumber \\
	                                  &= J
\end{align}
so that
\begin{align}
	\partial_{A^{\R}}\mathcal{L}-\nabla\paren{\partial_{\paren{\nabla A}^{\R}}\mathcal{L}} &= 
		J - \nabla\paren{\nabla \W A} = 0 \\
		\nabla F &= J
\end{align}




\subsection{Case 2 - The Dirac Field}
For the Dirac field
\begin{align}
		\mathcal{L} &= \grade{\nabla\psi I\gamma_{z}\psi^{\R}-eA\psi\gamma_{t}\psi^{\R}-m\psi\psi^{\R}}{} \nonumber \\
		            &= \paren{\nabla\psi}*\paren{I\gamma_{z}\psi^{\R}}-\grade{eA\psi\gamma_{t}\psi^{\R}}{} - m\psi*\psi^{\R} 
		               \label{eq8_110}\\
		            &= \paren{\nabla\psi I\gamma_{z}}*\psi^{\R}-\grade{eA\psi\gamma_{t}\psi^{\R}}{} - m\psi*\psi^{\R} \label{eq8_111}
\end{align}
The only term in eq~\ref{eq8_110} and eq~\ref{eq8_111} that we cannot immediately differentiate is 
$\partial_{\psi^{\R}}\grade{eA\psi\gamma_{t}\psi^{\R}}{}$. To perform this operation let $\psi^{\R} = X$ and use the definition of
the scalar directional derivative
\begin{align}
	\paren{B*\partial_{X}}\grade{eAX^{\R}\gamma_{t}X}{} &= 
		\lim_{h\rightarrow0}\bfrac{\grade{eA\paren{X^{\R}+hB^{\R}}\gamma_{t}\paren{X+hB}-eAX^{\R}\gamma_{t}X}{}}{h} \nonumber \\
		&= \grade{eAB^{\R}\gamma_{t}X+eAX^{\R}\gamma_{t}B}{} \nonumber \\
		&= B^{\R}*\paren{e\gamma_{t}XA}+B*\paren{eAX^{\R}\gamma_{t}} \nonumber \\
		&= B*\paren{eAX^{\R}\gamma_{t}}+B*\paren{eAX^{\R}\gamma_{t}} \nonumber \\
		&= B*\paren{2eAX^{\R}\gamma_{t}} \\
	 \partial_{X}\grade{eAX^{\R}\gamma_{t}X}{} &= 2eAX^{\R}\gamma_{t} \\
	 \partial_{\psi^{\R}}\grade{eA\psi\gamma_{t}\psi^{\R}}{} &= 2eA\psi\gamma_{t}
\end{align}
The other multivector derivatives are evaluated using the formulas in section~\ref{MV_derivatives}
\begin{align}
	\partial_{\paren{\nabla \psi}^{\R}} \paren{\paren{\nabla\psi}*\paren{I\gamma_{z}\psi^{\R}}} &=  \psi\gamma_{z}I^{\R} 
		= \psi\gamma_{z}I = -\psi I\gamma_{z} \\
	\partial_{\psi^{\R}}\paren{\paren{\nabla\psi I\gamma_{z}}*\psi^{\R}} &= \nabla\psi I\gamma_{z} \\
	\partial_{\psi^{\R}}\paren{m\psi*\psi^{\R}} &= 2m\psi.
\end{align}
The Lagrangian field equation are then
\begin{align}
	\partial_{\psi^{\R}}\mathcal{L}-\nabla\paren{\partial_{\paren{\nabla \psi}^{\R}}\mathcal{L}} &= \nabla\psi I\gamma_{z}
		-2eA\psi\gamma_{t}-2m\psi+\nabla\psi I\gamma_{z} = 0 \nonumber \\
		&= 2\paren{\nabla\psi I\gamma_{z}-eA\psi\gamma_{t}-m\psi} = 0 \nonumber \\
		0 &= \nabla\psi I\gamma_{z}-eA\psi\gamma_{t}-m\psi.
\end{align}

\subsection{Case 3 - The Coupled Electromagnetic and Dirac Fields}

For coupled electromagnetic and electron (Dirac) fields the coupled Lagrangian is 
\begin{align}
		\mathcal{L} &= \grade{\nabla\psi I\gamma_{z}\psi^{\R}-eA\psi\gamma_{t}\psi^{\R}-m\psi\psi^{\R}+\half F^{2}}{} \nonumber \\
		            &= \paren{\nabla\psi}*\paren{I\gamma_{z}\psi^{\R}}-\grade{eA\psi\gamma_{t}\psi^{\R}}{} - m\psi*\psi^{\R} 
		               +\half F^{2} \\
		            &= \paren{\nabla\psi I\gamma_{z}}*\psi^{\R}-eA*\paren{\psi\gamma_{t}\psi^{\R}} - m\psi*\psi^{\R} +\half F^{2},
\end{align}
where $A$ is the 4-vector potential dynamical field and $\psi$ is the spinor dynamical field. Between the previously calculated 
multivector derivatives for the electromagnetic and Dirac fields the only new derivative that needs to be calculated for the 
Lagrangian field equations is 
\begin{align}
	\partial_{A^{\R}}\paren{eA*\paren{\psi\gamma_{t}\psi^{\R}}} &= e\paren{\psi\gamma_{t}\psi^{\R}}^{\R} = e\psi\gamma_{t}\psi^{\R}.
\end{align}
The Lagrangian equations for the coupled fields are then
\begin{align}
	\partial_{\psi^{\R}}\mathcal{L} -\nabla\paren{\partial_{\paren{\nabla \psi}^{\R}}\mathcal{L}} &=
		2\paren{\nabla\psi I\gamma_{z}-eA\psi\gamma_{t}-m\psi} = 0 \\
	\partial_{A^{\R}}\mathcal{L} -\nabla\paren{\partial_{\paren{\nabla A}^{\R}}\mathcal{L}} &= 
		-e\psi\gamma_{t}\psi^{\R}+\nabla\paren{\nabla \W A} = 0
\end{align}
or
\begin{align}
	\nabla\psi I\gamma_{z}-eA\psi\gamma_{t} &= m\psi \\
	\nabla F = e\psi\gamma_{t}\psi^{\R}.
\end{align}