\chapter{Lie Groups}

Consider a 3-dimensional rotor $R$ where we have $RR^{\R}=1$. Then $R$ can be written
\be
    R = x_{0}+x_{1}Ie_{1}+x_{2}Ie_{2}+x_{2}Ie_{2}+x_{3}Ie_{3}
\ee
subject to the constraint
\be\label{eq8_2}
    RR^{\R} = x_{0}^{2}+x_{1}^{2}+x_{2}^{2}+x_{3}^{2} = 1.
\ee
Equation~\ref{eq8_2} defines a sphere in a 4-dimensional Euclidian space with each point on the sphere corresponding to a rotor.  This set
is called a 3-sphere denoted $S^{3}$. The surface $S^{3}$ is a manifold.

Since all rotations are generated by the formula $RaR^{\R}$, both $R$ and $-R$ generate the same rotation.  The group manifold for 3-dimensional
rotations is therefore more complicated than $S^{3}$ since opposite points on $S^{3}$ generate the same rotation. The group manifold is the 
appropriate setting for a Lagrangian treatment.  This has implications for constructing conjugate momenta, which are essential for the transition
to a quantum theory.  

\section{Definition of a Lie Group}

A Lie group is defined to be a manifold, $\cal{M}$, and a product $\f{\phi}{x,y}$ where $x,y \in \cal{M}$. Points on the manifold
can be labelled with vectors $\set{x,y}$, which can be viewed as lying in a higher dimensional embedding space (such as the case of
the 3-sphere in a 4-dimensional space and rotors in 3-space). 

The axioms of the group are
\begin{enumerate}
    \item {\em Closure.} $x\circ y \in {\cal M}\quad\forall x,y \in\cal{M}$.
    \item {\em Identity.} There exists an element $e \in {\cal M}$ such that $x\circ e = e\circ x = x,\quad \forall x\in{\cal M}$.
    \item {\em Inverse.} Fore every element $e \in {\cal M}$ there exists a unique element $\bar{e} \in {\cal M}$ such that 
    $x\circ\bar{x}=\bar{x}\circ x=e$.
    \item {\em Associativity.} $\paren{x\circ y}\circ z = x\circ\paren{y\circ z},\quad\forall x,y,z\in{\cal M}$.
\end{enumerate}
Any manifold with a product with the preceding properties is called a Lie group manifold.  Many of the group properties
of the group can be determined by examining the properties near the identity element.  The product then induces a 
{\em Lie bracket} structure on elements of the tangent space at the identity.  The tangent space is a linear space and the
vectors in this space together with their bracket form a Lie algebra.

\section{Spin Groups and the Bivector Algebra}

