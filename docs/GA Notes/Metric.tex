\documentclass[12pt]{book}
% setup the page geometry for landscape and use maximum screen real estate
%\usepackage[dvips]{geometry,graphicx,color}
\usepackage[pdftex]{geometry,graphicx,color}
\usepackage{bm}
\usepackage{amsmath}
\usepackage{tensor}
\usepackage{amsfonts}
\usepackage{comment}
\usepackage{srcltx}
\usepackage{amssymb}
\usepackage{rotating}
\usepackage{setspace}
\usepackage{parskip}
\usepackage{hyperref}
\geometry{headsep=2.0em,hscale=0.80}
\newcommand{\ubh}{\bm{\hat{u}}}
\newcommand{\ebh}{\bm{\hat{e}}}
\newcommand{\ebf}{\bm{e}}
%\newcommand{\ebh}{\hat{eb}}
%\definecolor{red}{rgb}{1,0,0}
\setlength{\parindent}{0in}
\usepackage{sectsty}
\sectionfont{\large}
%\titlespacing*{\section}{0pt}{6pt}{3pt}
\newcommand{\ds}{\displaystyle}
\newcommand{\be}{\begin{equation}}
%\newcommand{\be}{\begin{math}}
\newcommand{\benn}{\begin{center}\begin{math}}
\newcommand{\ee}{\end{equation}}
%\newcommand{\ee}{\end{math}}
\newcommand{\eenn}{\end{math}\end{center}}
\newcommand{\lp}{\left (}
\newcommand{\rp}{\right )}
\newcommand{\lmat}{\left [}
\newcommand{\rmat}{\right ]}
\newcommand{\mat}[1]{\lmat {#1}\rmat}
\newcommand{\lb}{\left \{}
\newcommand{\rb}{\right \}}
\newcommand{\w}{\wedge}
\newcommand{\bfrac}[2]{\displaystyle\frac{#1}{#2}}
\newcommand{\half}{\bfrac{1}{2}}
\newcommand{\wprod}[2]{{#1}\w\dots\w{#2}}
\newcommand{\vprod}[2]{{#1}\dots{#2}}
\newcommand{\Sum}[2]{\sum_{{#1},\dots,{#2}}}
\newcommand{\paren}[1]{\lp {#1} \rp}
\newcommand{\lst}[2]{{#1},\dots,{#2}}
\newcommand{\wprodi}[3]{{#1}\w\dots\w{#2}\w\dots\w{#3}}
\newcommand{\mset}[2]{{#1},\dots,{#2}}
\newcommand{\proj}[2]{\left \langle {#1}\right \rangle_{#2}}
\newcommand{\abs}[1]{\left | {#1} \right |}
\newcommand{\tanhf}[1]{\tanh\lp{#1}\rp}
\newcommand{\thh}{\frac{\theta}{2}}
\newcommand{\eperp}{e_{\perp}}
\newcommand{\epar}{e_{\parallel}}
\newcommand{\epep}{\eperp\epar}
\newcommand{\epepr}{\epar\eperp}
\newcommand{\vp}{v_{\parallel}}
\newcommand{\fof}[2]{{#1}\paren{#2}}
\newcommand{\R}{\dagger}
\newcommand{\GA}[1]{\mathcal{G}\paren{#1}}
\newcommand{\VS}[1]{\mathcal{V}\paren{#1}}
\newcommand{\adj}[1]{\overline{#1}}
\newcommand{\Mder}[3]{\f{\underline{#1}_{#2}}{#3}}
\newcommand{\as}[1]{\frac{{#1}(#1-1)}{2}}
\newcommand{\onep}[1]{\paren{-1}^{#1}}
\newcommand{\adjf}[2]{\overline{#1}\paren{#2}}
\newcommand{\Vsp}{\mathcal{V}}
\newcommand{\uv}[1]{\widehat{\mathbf {#1}}}
\newcommand{\ebar}{\bar{e}}
\newcommand{\nbar}{\bar{n}}
\newcommand{\h}[1]{\frac{#1}{2}}
\newcommand{\Uh}{\widehat{U}}
\newcommand{\brkt}[1]{\left \{ {#1} \right \} }
\newcommand{\eb}{\bm{e}}
\newcommand{\rv}{{\bf r}}
\newcommand{\cross}{{\mathbf \times}}
\newcommand{\bv}[1]{{\mathbf {#1}}}
\newcommand{\xv}{\bv{x}}
\newcommand{\fv}{\bv{f}}
\newcommand{\pv}{\bv{p}}
\newcommand{\ev}{\bv{e}}
\newcommand{\vv}{\bv{v}}
\newcommand{\Xv}{\bv{X}}
\newcommand{\Pv}{\bv{P}}
\newcommand{\W}{\wedge}
\newcommand{\ddt}[1]{\bfrac{d{#1}}{dt}}
\newcommand{\Dt}[1]{\dot{#1}}
\newcommand{\DDt}[1]{\ddot{#1}}
\newcommand{\grad}{{\mathbf \nabla}}
\newcommand{\gO}{\gamma_{0}}
\newcommand{\gl}{\gamma_{1}}
\newcommand{\alphah}{\frac{\alpha}{2}}
\newcommand{\rhat}{\hat{r}}
\newcommand{\fr}{\fof{f}{r}}
\newcommand{\frhlf}{\bfrac{\fof{f}{r}}{2}}
\newcommand{\Xdotn}{\lp X\cdot n \rp}
\newcommand{\Ydotn}{\lp Y\cdot n \rp}
\newcommand{\Xdote}{\lp X\cdot e \rp}
\newcommand{\Ydote}{\lp Y\cdot e \rp}
\newcommand{\XdotY}{\lp X\cdot Y \rp}
\newcommand{\csh}[1]{\cosh\lp{#1}\rp}
\newcommand{\snh}[1]{\sinh\lp{#1}\rp}
\newcommand{\acsh}[1]{\Cosh^{-1}\lp{#1}\rp}
\newcommand{\asnh}[1]{\Sinh^{-1}\lp{#1}\rp}
\newcommand{\Bhat}{\hat{B}}
\newcommand{\e}{\mbox{e}}
\newcommand{\pdiff}[2]{\bfrac{\partial{#1}}{\partial{#2}}}
\newcommand{\deriv}[2]{\bfrac{d{#1}}{d{#2}}}
\newcommand{\derivn}[3]{\bfrac{d^{#3}{#1}}{d{#2}^{#3}}}
\newcommand{\ldi}[1]{\lambda^{#1}}
\newcommand{\Ldi}[1]{\Lambda_{#1}}
\newcommand{\dldi}[1]{d\lambda^{#1}}
\newcommand{\chn}[2]{\paren{#1}_{\paren{#2}}}
\newcommand{\Aijk}[2]{A_{#1}^{\paren{#2}}}
\newcommand{\Bijk}[2]{B_{#1}^{\paren{#2}}}
\newcommand{\eabs}[1]{\left | \bm{e}_{#1} \right |}
\newcommand{\ebl}[1]{\bm{e}_{\left [ {#1} \right ]}}
\newcommand{\eblh}[1]{\hat{\bm{e}}_{\left [ {#1} \right ]}}
\newcommand{\Gh}[2]{\hat{\Gamma}_{#1}^{#2}}
\newcommand{\fct}[2]{{#1}\!\lp{#2}\rp}
\newcommand{\grd}[2]{\left \langle {#1} \right \rangle_{#2}}
\newcommand{\ub}{\bm{u}}
\newcommand{\cosf}[1]{\operatorname{cos}\lp{#1}\rp}
\newcommand{\sinf}[1]{\operatorname{sin}\lp{#1}\rp}
\newcommand{\coshf}[1]{\operatorname{cosh}\lp{#1}\rp}
\newcommand{\sinhf}[1]{\operatorname{sinh}\lp{#1}\rp}
\newcommand{\smplx}[2]{\lp {#1} \rp_{\lp {#2} \rp}}
\newcommand{\f}[2]{{#1}\lp {#2} \rp}
\newcommand{\Llin}{\mathsf{L}}
\newcommand{\grdO}[1]{\left\langle{#1}\right\rangle}
\newcommand{\limeps}{\lim_{\epsilon\to 0}}
\newcommand{\grade}[2]{\left < {#1} \right >_{#2}}
\newcommand{\set}[1]{\lb {#1} \rb}
\newcommand{\prj}{\mathcal{P}}
\newcommand{\prjp}{\mathcal{P}_{\perp}}
\newcommand{\Prj}[1]{\f{\prj}{#1}}
\newcommand{\Prjprm}[1]{\f{\prj'}{#1}}
\newcommand{\Prjp}[1]{\f{\prjp}{#1}}
\newcommand{\braket}[1]{\left < {#1} \right > }
\newcommand{\cbrk}[2]{\left [ {#1},{#2}\right ]}
\newcommand{\Mat}[1]{\left [ {#1} \right ]}
\newcommand{\acbrk}[2]{\left \{ {#1},{#2}\right \}}
\newcommand{\Set}[1]{\left \{ {#1} \right \}}
\newcommand{\rten}{\mathcal{R}}
\newcommand{\Rten}[1]{\f{\rten}{#1}}
\newcommand{\sff}{\mathsf{f}}
\newcommand{\Lie}[2]{\mathcal{L}_{#1}{#2}}
\newcommand{\Liep}[2]{\mathcal{L'}_{{#1}'}{{#2}'}}
\newcommand{\Wbld}{\hat{\W}}
\newcommand{\cdotbld}{\boldsymbol{\cdot}}
\newcommand{\eval}[2]{\left . {#1} \right |_{#2}}
\newcommand{\cosp}[1]{\cos{\paren{{#1}}}}
\newcommand{\sinp}[1]{\sin{\paren{{#1}}}}
\newcommand{\cosr}{\cosp{\bfrac{\rho}{r}}}
\newcommand{\sinr}{\sinp{\bfrac{\rho}{r}}}
\newcommand{\Eval}[2]{\left . {#1}\right |_{#2}}
\newcommand{\Pval}[2]{\lp {#1}\rp_{#2}}
\newcommand{\ndx}[2]{{#1}_{\left \{{#2}\right \}}}
\newcommand{\eu}[2]{e^{\ndx{#1}{#2}}}
\newcommand{\ed}[2]{e_{\ndx{#1}{#2}}}
\newcommand{\fbld}{\mathbf{f}}
\newcommand{\var}[1]{\delta\hspace{-2pt}{#1}}
\newcommand{\lgrad}{\overleftarrow{\nabla}}
\newcommand{\Lf}{\mathcal{L}}
\newcommand{\Ev}{\vec{E}}
\newcommand{\Jv}{\vec{J}}
\newcommand{\Bv}{\vec{B}}
\newcommand{\Fv}{\vec{F}}
\newcommand{\g}[1]{\gamma_{#1}}
\newcommand{\gr}[1]{\gamma^{#1}}
\newcommand{\Sig}[1]{\vec{\sigma}_{#1}}
\newcommand{\Nabla}{\vec{\nabla}}
\newcommand{\pd}[1]{\partial_{#1}}
\newcommand{\gu}{\underline{g}}
\newcommand{\go}{\overline{g}}
\newcommand{\alphau}{\underline{\alpha}}
\newcommand{\alphao}{\overline{\alpha}}
\newcommand{\Gu}{\underline{G}}
\newcommand{\Go}{\overline{G}}
\newcommand{\nn}{\nonumber}
\newcommand{\ovl}[1]{\overline{#1}}
\newcommand{\pD}[2]{\partial_{#1}{#2}}
\newcommand{\Bra}[1]{\left < {#1} \right |}
\newcommand{\Ket}[1]{\left | {#1} \right >}
\newcommand{\BraKet}[3]{\left < {#1}\left | {#2} \right | {#3} \right >}
\newcommand{\Braket}[2]{\left < {#1}\left | \right . {#2} \right >}
\newcommand{\PB}[2]{\left [ {#1} \right ]_{#2}}
\newcommand{\avg}[1]{\left < {#1} \right >}
\newcommand{\inner}[2]{\left < {#1},{#2} \right >}
\newcommand{\Inner}[2]{\left < {#1}\left | {#2} \right .\right >}
\newcommand{\ubar}[1]{\underline{#1}}
%\newcommand{\half}{\bfrac{1}{2}}
\newcommand{\ebb}{\bar{\bm{e}}}
\newcommand{\wb}{\bm{w}}
\newcommand{\SO}[1]{\f{\mbox{SO}\!}{#1}}
\newcommand{\GL}[1]{\f{\mbox{GL}\!}{#1}}
\newcommand{\so}[1]{\f{\mathfrak{so}\!}{#1}}
\newcommand{\lgl}[1]{\f{\mathfrak{gl}\!}{#1}}
\newcommand{\Spin}[1]{\f{\mbox{Spin}\!}{#1}}
\newcommand{\shalf}{\frac{1}{2}}
\newcommand{\edot}{\bm{:}}
\newcommand{\alphab}{\bm{\alpha}}
\newcommand{\betab}{\bm{\beta}}
\newcommand{\braces}[1]{\b{#1}\rb}
\newcommand{\bracetwo}[2]{\lb\begin{array}{c}{#1}\\{#2}\end{array}\rb}
\newcommand{\mfrk}[1]{\mathfrak{#1}}
%\newcommand{\ub}{\bm{u}}
\newcommand{\vb}{\bm{v}}
\newcommand{\isur}{\mathcal{i}}
\newcommand{\End}{\mathcal{E}nd}
\newcommand{\Aut}{\mathcal{A}ut}
\title{\bf\Large Manipulations of the Metric}
\author{\bf Alan Bromborsky\\
\bf Army Research Lab (Retired)\\
\bf abrombo@verizon.net}

\begin{document}
\parskip 10pt
\maketitle
Let us consider three different ways of realizing a manifold which I call ``implicit'', ``derived'', and 
``explicit''.

The ``implicit'' manifold is defined by a set of coordinates $\bm{\theta} = \set{\theta_{1},\dots,\theta_{n}}$, 
a set of basis vectors $\set{\eb_{\theta_{1}},\dots,\eb_{\theta_{n}}}$, 
and a metric tensor $\f{g_{ij}}{\bm{\theta}}$ where the
metric tensor is related to the basis vectors by 
\be
	g_{ij} = \eb_{\theta_{i}}\cdot\eb_{\theta_{j}},
\ee
and the $\eb_{\theta_{i}}$ are implicitly a function of the coordinates.  That is to say for an ``implicit'' manifold
the metric tensor defines the dot products of the basis vectors and not vice versa. This way of defining a 
manifold is usually encountered in general relativity.

The ``derived'' manifold is given by a single vector field, $\f{\bm{X}}{\bm{\theta}}$, that is a function of 
the coordinates that defined in some embedding vector space. In this case the basis vectors, 
$\set{\eb_{\theta_{1}},\dots,\eb_{\theta_{n}}}$ and the metric tensor, $\f{g_{ij}}{\bm{\theta}}$, are derived
from $\f{\bm{X}}{\bm{\theta}}$ via
\begin{align}
	\eb_{\theta_{i}} &= \pdiff{\bm{X}}{\theta_{i}} \\
	g_{ij} &= \eb_{\theta_{i}}\cdot\eb_{\theta_{j}},
\end{align}
where the dot product is that of the embedding vector space. In this case the basis vectors and the metric tensor
are derived from $\f{\bm{X}}{\bm{\theta}}$. An example of this would be a two dimensional manifold embedded in
a three dimensional space such as a sphere defined by
$$\f{\bm{X}}{\theta,\phi} = 
\f{\cos}{\theta}\eb_{z}+\f{\sin}{\theta}\paren{\f{\cos}{\phi}\eb_{x}+\f{\sin}{\phi}\eb_{y}}.$$

The ``explicit'' manifold is one in which the basis vectors are given vector fields, 
$\f{\eb_{\theta_{i}}}{\bm{\theta}}$, in an embedding vector space and the metric tensor is calculated as in the
case of the ``derived'' manifold. An example of this would be the spherical coordinate basis vectors
\begin{align*}
	\eb_{r} &= \f{\cos}{\theta}\eb_{z}+\f{\sin}{\theta}\paren{\f{\cos}{\phi}\eb_{x}+\f{\sin}{\phi}\eb_{y}}, \\
	\eb_{\theta} &= \f{\sin}{\theta}\eb_{z}-\f{\cos}{\theta}\paren{\f{\cos}{\phi}\eb_{x}+\f{\sin}{\phi}\eb_{y}}, \\
	\eb_{\phi} &= \f{-\sin}{\phi}\eb_{x}+\f{\cos}{\phi}\eb_{y}.
\end{align*}
In this case the basis vectors can be normalized vector fields and the metric tensor is calculated as in the
``derived'' case.

In both the ``derived'' and ``explicit'' cases the derivatives of the basis vectors can be calculated by 
direct differentiation of the $\f{\eb_{\theta_{i}}}{\bm{\theta}}$'s. However, in the case of the ``implicit'' manifold
we must proceed as follows.

The derivatives of the basis vectors are calculated from the Christ-Awful symbols, $\f{\Gamma_{ij}^{k}}{\bm{\theta}}$,
by (Einstein summation convention used)
\be
	\pdiff{\eb_{\theta_{i}}}{\theta_{j}} = \f{\Gamma_{ij}^{k}}{\bm{\theta}}\eb_{\theta_{k}}.
\ee
Where the $\f{\Gamma_{ij}^{k}}{\bm{\theta}}$ are derived from the $\f{g_{ij}}{\bm{\theta}}$.

Now let us derive all the relevant manifold quatities in terms of normalized basis vectors.

We define a norm for the basis vectors by
\be
	\abs{\eb_{\theta_{i}}} \equiv \sqrt{\abs{g_{ii}}}
\ee
and
\be
	\eb_{\theta_{i}} = \abs{\eb_{\theta_{i}}}\ebh_{\theta_{i}}
\ee
so that $\ebh_{\theta_{i}}\cdot\ebh_{\theta_{i}} = \pm 1$.

Now consider what happens if we use a normalized basis $\set{\ebh_{\theta_{1}},\dots,\ebh_{\theta_{n}}}$.
\begin{align}
	\pdiff{\eb_{\theta_{i}}}{\theta_{j}} &= 
		\pdiff{}{\theta_{j}}\paren{\abs{\eb_{\theta_{i}}}\ebh_{\theta_{i}}} \nonumber \\
		& = \pdiff{\abs{\eb_{\theta_{i}}}}{\theta_{j}}\ebh_{\theta_{i}}
		    +\abs{\eb_{\theta_{i}}}\pdiff{\ebh_{\theta_{i}}}{\theta_{j}} \nonumber \\
	\pdiff{\ebh_{\theta_{i}}}{\theta_{j}} &=
			\bfrac{1}{\abs{\eb_{\theta_{i}}}}\paren{\pdiff{\eb_{\theta_{i}}}{\theta_{j}}
			- \pdiff{\abs{\eb_{\theta_{i}}}}{\theta_{j}}\ebh_{\theta_{i}}},
\end{align}
but
\begin{align}
	\pdiff{\abs{\eb_{\theta_{i}}}^{2}}{\theta_{j}} &= 
		2\abs{\eb_{\theta_{i}}}\pdiff{\abs{\eb_{\theta_{i}}}}{\theta_{j}} \nonumber \\
	\pdiff{\abs{\eb_{\theta_{i}}}}{\theta_{j}} &= \bfrac{1}{2\abs{\eb_{\theta_{i}}}}
	                                              \pdiff{\abs{\eb_{\theta_{i}}}^{2}}{\theta_{j}} \nonumber \\
	                                           &= \bfrac{\abs{\eb_{\theta_{i}}}}{2\abs{g_{ii}}}
	                                              \pdiff{\abs{g_{ii}}}{\theta_{j}}.
\end{align}
So that
\begin{align}
	\pdiff{\ebh_{\theta_{i}}}{\theta_{j}} &= 
		\bfrac{1}{\abs{\eb_{\theta_{i}}}}\pdiff{\eb_{\theta_{i}}}{\theta_{j}}
		- \bfrac{1}{2\abs{g_{ii}}}\pdiff{\abs{g_{ii}}}{\theta_{j}}\ebh_{\theta_{i}} \nonumber \\
	    &= \bfrac{\abs{\eb_{\theta_{k}}}}{\abs{\eb_{\theta_{i}}}}\Gamma_{ij}^{k}\ebh_{\theta_{k}}
	      - \bfrac{1}{2\abs{g_{ii}}}\pdiff{\abs{g_{ii}}}{\theta_{j}}\ebh_{\theta_{i}}.
\end{align}
Note that the derivatives of the normalized basis vectors are not even in the same direction as the derivatives
of the original basis vectors.

My thesis is if the $\eb_{\theta_{i}}$'s are normalized you cannot calculate the 
$\pdiff{\eb_{\theta_{i}}}{\theta_{j}}$'s unless you explicitly know the
 $\f{\eb_{\theta_{i}}}{\bm{\theta}}$ as vector fields in an embedding vector space.  This can be done for
``derived'' and ``explicit'' manifolds, but not for ``implicit'' manifolds. For ``implicit'' manifolds you can 
derive a set of normalized basis vectors and their derivatives in terms of the normalized basis vectors from
a metric for unnormalized basis vectors as shown above.
 
An exception to my thesis could be a case of non-orthogonal basis vectors for then the off diagonal elements of
the metric tensor could encode the necessary information or again maybe not?  

\end{document}
